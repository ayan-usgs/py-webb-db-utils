% Generated by Sphinx.
\def\sphinxdocclass{report}
\documentclass[letterpaper,10pt,english]{sphinxmanual}
\usepackage[utf8]{inputenc}
\DeclareUnicodeCharacter{00A0}{\nobreakspace}
\usepackage{cmap}
\usepackage[T1]{fontenc}
\usepackage{babel}
\usepackage{times}
\usepackage[Bjarne]{fncychap}
\usepackage{longtable}
\usepackage{sphinx}
\usepackage{multirow}


\title{py-webb-db-utils Documentation}
\date{September 08, 2014}
\release{a1}
\author{Andrew Yan}
\newcommand{\sphinxlogo}{}
\renewcommand{\releasename}{Release}
\makeindex

\makeatletter
\def\PYG@reset{\let\PYG@it=\relax \let\PYG@bf=\relax%
    \let\PYG@ul=\relax \let\PYG@tc=\relax%
    \let\PYG@bc=\relax \let\PYG@ff=\relax}
\def\PYG@tok#1{\csname PYG@tok@#1\endcsname}
\def\PYG@toks#1+{\ifx\relax#1\empty\else%
    \PYG@tok{#1}\expandafter\PYG@toks\fi}
\def\PYG@do#1{\PYG@bc{\PYG@tc{\PYG@ul{%
    \PYG@it{\PYG@bf{\PYG@ff{#1}}}}}}}
\def\PYG#1#2{\PYG@reset\PYG@toks#1+\relax+\PYG@do{#2}}

\expandafter\def\csname PYG@tok@gd\endcsname{\def\PYG@tc##1{\textcolor[rgb]{0.63,0.00,0.00}{##1}}}
\expandafter\def\csname PYG@tok@gu\endcsname{\let\PYG@bf=\textbf\def\PYG@tc##1{\textcolor[rgb]{0.50,0.00,0.50}{##1}}}
\expandafter\def\csname PYG@tok@gt\endcsname{\def\PYG@tc##1{\textcolor[rgb]{0.00,0.27,0.87}{##1}}}
\expandafter\def\csname PYG@tok@gs\endcsname{\let\PYG@bf=\textbf}
\expandafter\def\csname PYG@tok@gr\endcsname{\def\PYG@tc##1{\textcolor[rgb]{1.00,0.00,0.00}{##1}}}
\expandafter\def\csname PYG@tok@cm\endcsname{\let\PYG@it=\textit\def\PYG@tc##1{\textcolor[rgb]{0.25,0.50,0.56}{##1}}}
\expandafter\def\csname PYG@tok@vg\endcsname{\def\PYG@tc##1{\textcolor[rgb]{0.73,0.38,0.84}{##1}}}
\expandafter\def\csname PYG@tok@m\endcsname{\def\PYG@tc##1{\textcolor[rgb]{0.13,0.50,0.31}{##1}}}
\expandafter\def\csname PYG@tok@mh\endcsname{\def\PYG@tc##1{\textcolor[rgb]{0.13,0.50,0.31}{##1}}}
\expandafter\def\csname PYG@tok@cs\endcsname{\def\PYG@tc##1{\textcolor[rgb]{0.25,0.50,0.56}{##1}}\def\PYG@bc##1{\setlength{\fboxsep}{0pt}\colorbox[rgb]{1.00,0.94,0.94}{\strut ##1}}}
\expandafter\def\csname PYG@tok@ge\endcsname{\let\PYG@it=\textit}
\expandafter\def\csname PYG@tok@vc\endcsname{\def\PYG@tc##1{\textcolor[rgb]{0.73,0.38,0.84}{##1}}}
\expandafter\def\csname PYG@tok@il\endcsname{\def\PYG@tc##1{\textcolor[rgb]{0.13,0.50,0.31}{##1}}}
\expandafter\def\csname PYG@tok@go\endcsname{\def\PYG@tc##1{\textcolor[rgb]{0.20,0.20,0.20}{##1}}}
\expandafter\def\csname PYG@tok@cp\endcsname{\def\PYG@tc##1{\textcolor[rgb]{0.00,0.44,0.13}{##1}}}
\expandafter\def\csname PYG@tok@gi\endcsname{\def\PYG@tc##1{\textcolor[rgb]{0.00,0.63,0.00}{##1}}}
\expandafter\def\csname PYG@tok@gh\endcsname{\let\PYG@bf=\textbf\def\PYG@tc##1{\textcolor[rgb]{0.00,0.00,0.50}{##1}}}
\expandafter\def\csname PYG@tok@ni\endcsname{\let\PYG@bf=\textbf\def\PYG@tc##1{\textcolor[rgb]{0.84,0.33,0.22}{##1}}}
\expandafter\def\csname PYG@tok@nl\endcsname{\let\PYG@bf=\textbf\def\PYG@tc##1{\textcolor[rgb]{0.00,0.13,0.44}{##1}}}
\expandafter\def\csname PYG@tok@nn\endcsname{\let\PYG@bf=\textbf\def\PYG@tc##1{\textcolor[rgb]{0.05,0.52,0.71}{##1}}}
\expandafter\def\csname PYG@tok@no\endcsname{\def\PYG@tc##1{\textcolor[rgb]{0.38,0.68,0.84}{##1}}}
\expandafter\def\csname PYG@tok@na\endcsname{\def\PYG@tc##1{\textcolor[rgb]{0.25,0.44,0.63}{##1}}}
\expandafter\def\csname PYG@tok@nb\endcsname{\def\PYG@tc##1{\textcolor[rgb]{0.00,0.44,0.13}{##1}}}
\expandafter\def\csname PYG@tok@nc\endcsname{\let\PYG@bf=\textbf\def\PYG@tc##1{\textcolor[rgb]{0.05,0.52,0.71}{##1}}}
\expandafter\def\csname PYG@tok@nd\endcsname{\let\PYG@bf=\textbf\def\PYG@tc##1{\textcolor[rgb]{0.33,0.33,0.33}{##1}}}
\expandafter\def\csname PYG@tok@ne\endcsname{\def\PYG@tc##1{\textcolor[rgb]{0.00,0.44,0.13}{##1}}}
\expandafter\def\csname PYG@tok@nf\endcsname{\def\PYG@tc##1{\textcolor[rgb]{0.02,0.16,0.49}{##1}}}
\expandafter\def\csname PYG@tok@si\endcsname{\let\PYG@it=\textit\def\PYG@tc##1{\textcolor[rgb]{0.44,0.63,0.82}{##1}}}
\expandafter\def\csname PYG@tok@s2\endcsname{\def\PYG@tc##1{\textcolor[rgb]{0.25,0.44,0.63}{##1}}}
\expandafter\def\csname PYG@tok@vi\endcsname{\def\PYG@tc##1{\textcolor[rgb]{0.73,0.38,0.84}{##1}}}
\expandafter\def\csname PYG@tok@nt\endcsname{\let\PYG@bf=\textbf\def\PYG@tc##1{\textcolor[rgb]{0.02,0.16,0.45}{##1}}}
\expandafter\def\csname PYG@tok@nv\endcsname{\def\PYG@tc##1{\textcolor[rgb]{0.73,0.38,0.84}{##1}}}
\expandafter\def\csname PYG@tok@s1\endcsname{\def\PYG@tc##1{\textcolor[rgb]{0.25,0.44,0.63}{##1}}}
\expandafter\def\csname PYG@tok@gp\endcsname{\let\PYG@bf=\textbf\def\PYG@tc##1{\textcolor[rgb]{0.78,0.36,0.04}{##1}}}
\expandafter\def\csname PYG@tok@sh\endcsname{\def\PYG@tc##1{\textcolor[rgb]{0.25,0.44,0.63}{##1}}}
\expandafter\def\csname PYG@tok@ow\endcsname{\let\PYG@bf=\textbf\def\PYG@tc##1{\textcolor[rgb]{0.00,0.44,0.13}{##1}}}
\expandafter\def\csname PYG@tok@sx\endcsname{\def\PYG@tc##1{\textcolor[rgb]{0.78,0.36,0.04}{##1}}}
\expandafter\def\csname PYG@tok@bp\endcsname{\def\PYG@tc##1{\textcolor[rgb]{0.00,0.44,0.13}{##1}}}
\expandafter\def\csname PYG@tok@c1\endcsname{\let\PYG@it=\textit\def\PYG@tc##1{\textcolor[rgb]{0.25,0.50,0.56}{##1}}}
\expandafter\def\csname PYG@tok@kc\endcsname{\let\PYG@bf=\textbf\def\PYG@tc##1{\textcolor[rgb]{0.00,0.44,0.13}{##1}}}
\expandafter\def\csname PYG@tok@c\endcsname{\let\PYG@it=\textit\def\PYG@tc##1{\textcolor[rgb]{0.25,0.50,0.56}{##1}}}
\expandafter\def\csname PYG@tok@mf\endcsname{\def\PYG@tc##1{\textcolor[rgb]{0.13,0.50,0.31}{##1}}}
\expandafter\def\csname PYG@tok@err\endcsname{\def\PYG@bc##1{\setlength{\fboxsep}{0pt}\fcolorbox[rgb]{1.00,0.00,0.00}{1,1,1}{\strut ##1}}}
\expandafter\def\csname PYG@tok@kd\endcsname{\let\PYG@bf=\textbf\def\PYG@tc##1{\textcolor[rgb]{0.00,0.44,0.13}{##1}}}
\expandafter\def\csname PYG@tok@ss\endcsname{\def\PYG@tc##1{\textcolor[rgb]{0.32,0.47,0.09}{##1}}}
\expandafter\def\csname PYG@tok@sr\endcsname{\def\PYG@tc##1{\textcolor[rgb]{0.14,0.33,0.53}{##1}}}
\expandafter\def\csname PYG@tok@mo\endcsname{\def\PYG@tc##1{\textcolor[rgb]{0.13,0.50,0.31}{##1}}}
\expandafter\def\csname PYG@tok@mi\endcsname{\def\PYG@tc##1{\textcolor[rgb]{0.13,0.50,0.31}{##1}}}
\expandafter\def\csname PYG@tok@kn\endcsname{\let\PYG@bf=\textbf\def\PYG@tc##1{\textcolor[rgb]{0.00,0.44,0.13}{##1}}}
\expandafter\def\csname PYG@tok@o\endcsname{\def\PYG@tc##1{\textcolor[rgb]{0.40,0.40,0.40}{##1}}}
\expandafter\def\csname PYG@tok@kr\endcsname{\let\PYG@bf=\textbf\def\PYG@tc##1{\textcolor[rgb]{0.00,0.44,0.13}{##1}}}
\expandafter\def\csname PYG@tok@s\endcsname{\def\PYG@tc##1{\textcolor[rgb]{0.25,0.44,0.63}{##1}}}
\expandafter\def\csname PYG@tok@kp\endcsname{\def\PYG@tc##1{\textcolor[rgb]{0.00,0.44,0.13}{##1}}}
\expandafter\def\csname PYG@tok@w\endcsname{\def\PYG@tc##1{\textcolor[rgb]{0.73,0.73,0.73}{##1}}}
\expandafter\def\csname PYG@tok@kt\endcsname{\def\PYG@tc##1{\textcolor[rgb]{0.56,0.13,0.00}{##1}}}
\expandafter\def\csname PYG@tok@sc\endcsname{\def\PYG@tc##1{\textcolor[rgb]{0.25,0.44,0.63}{##1}}}
\expandafter\def\csname PYG@tok@sb\endcsname{\def\PYG@tc##1{\textcolor[rgb]{0.25,0.44,0.63}{##1}}}
\expandafter\def\csname PYG@tok@k\endcsname{\let\PYG@bf=\textbf\def\PYG@tc##1{\textcolor[rgb]{0.00,0.44,0.13}{##1}}}
\expandafter\def\csname PYG@tok@se\endcsname{\let\PYG@bf=\textbf\def\PYG@tc##1{\textcolor[rgb]{0.25,0.44,0.63}{##1}}}
\expandafter\def\csname PYG@tok@sd\endcsname{\let\PYG@it=\textit\def\PYG@tc##1{\textcolor[rgb]{0.25,0.44,0.63}{##1}}}

\def\PYGZbs{\char`\\}
\def\PYGZus{\char`\_}
\def\PYGZob{\char`\{}
\def\PYGZcb{\char`\}}
\def\PYGZca{\char`\^}
\def\PYGZam{\char`\&}
\def\PYGZlt{\char`\<}
\def\PYGZgt{\char`\>}
\def\PYGZsh{\char`\#}
\def\PYGZpc{\char`\%}
\def\PYGZdl{\char`\$}
\def\PYGZhy{\char`\-}
\def\PYGZsq{\char`\'}
\def\PYGZdq{\char`\"}
\def\PYGZti{\char`\~}
% for compatibility with earlier versions
\def\PYGZat{@}
\def\PYGZlb{[}
\def\PYGZrb{]}
\makeatother

\renewcommand\PYGZsq{\textquotesingle}

\begin{document}

\maketitle
\tableofcontents
\phantomsection\label{webb_utils_doc::doc}


Contents:


\chapter{Package Installation}
\label{installation::doc}\label{installation:welcome-to-py-webb-db-utils-s-documentation}\label{installation:package-installation}
If not already installed, you will want to install Python 2.7. Downloads can be found
at \href{https://www.python.org/download/releases/}{https://www.python.org/download/releases/}. The 32-bit verson of Python is recommended
as it is better supported by the Python community.

A C complier may also need to be installed and setup. This may take of form of GCC,
MS Visual Studio 2008, or MinGW32.

There are a number of Python package dependencies that are required for the running
of this package. These dependencies are summarized in requirements.txt and are summarized
here for convenience. All of these packages may be installed using pip from the command line
(e.g. \code{pip install python-dateutil}).
\begin{itemize}
\item {} 
SQLAlchemy==0.9.7

\item {} 
cx-Oracle==5.1.3

\item {} 
numpy==1.8.1

\item {} 
pandas==0.14.1

\item {} 
python-dateutil==2.2

\item {} 
pytz==2014.4

\item {} 
six==1.7.3

\end{itemize}

Some of these dependencies may prove difficult to install on Windows. If this is a problem, unofficial
binaries of these packages may be found from the University of California, Irvine at
\href{http://www.lfd.uci.edu/~gohlke/pythonlibs/}{http://www.lfd.uci.edu/\textasciitilde{}gohlke/pythonlibs/}.

Once dependenacies are installed, the package may be installed can be installed through the following steps:
\begin{itemize}
\item {} 
In the command line, change to the py-webb-db-utils directory

\item {} 
Run the following command: \code{python setup.py install}

\end{itemize}


\chapter{Description of Package Modules}
\label{modules::doc}\label{modules:description-of-package-modules}

\section{Base SQL}
\label{modules:base-sql}
Module containing the SQL used in the database queries.
Should the SQL need to be changed, this is the place to
do it.


\section{Sites}
\label{modules:sites}
List of sites and site groups. This sites may be changed
if a new site is added to the WEBB project.


\section{Retrieve Data}
\label{modules:retrieve-data}
Module containing classes for creating objects to
effect data retrieval from an Oracle database.
\index{RetrieveData (class in webb\_utils.retrieve\_data)}

\begin{fulllineitems}
\phantomsection\label{modules:webb_utils.retrieve_data.RetrieveData}\pysiglinewithargsret{\strong{class }\code{webb\_utils.retrieve\_data.}\bfcode{RetrieveData}}{\emph{schema}, \emph{password}, \emph{db\_name}, \emph{excel\_indexes=False}}{}
Retrieve data object to execute and retrieve the results
of six database queries deemed important to the USGS WEBB
project.
\begin{quote}\begin{description}
\item[{Parameters}] \leavevmode\begin{itemize}
\item {} 
\textbf{schema} (\emph{str}) -- schema user name

\item {} 
\textbf{password} (\emph{str}) -- schema user password

\item {} 
\textbf{db\_name} (\emph{str}) -- database name

\item {} 
\textbf{excel\_indexes} (\emph{bool}) -- whether an exported excel file should have Pandas dataframe indexes; default is False

\end{itemize}

\end{description}\end{quote}
\index{\_create\_dataframe() (webb\_utils.retrieve\_data.RetrieveData method)}

\begin{fulllineitems}
\phantomsection\label{modules:webb_utils.retrieve_data.RetrieveData._create_dataframe}\pysiglinewithargsret{\bfcode{\_create\_dataframe}}{\emph{data}, \emph{columns}}{}
Internal method to create a pandas dataframe.
\begin{quote}\begin{description}
\item[{Parameters}] \leavevmode\begin{itemize}
\item {} 
\textbf{data} (\emph{list of tuples}) -- raw query results

\item {} 
\textbf{columns} (\emph{list of strings}) -- dataframe column names

\end{itemize}

\item[{Returns}] \leavevmode
dataframe of query results

\item[{Return type}] \leavevmode
pandas.DataFrame

\end{description}\end{quote}

\end{fulllineitems}

\index{get\_carbon\_data() (webb\_utils.retrieve\_data.RetrieveData method)}

\begin{fulllineitems}
\phantomsection\label{modules:webb_utils.retrieve_data.RetrieveData.get_carbon_data}\pysiglinewithargsret{\bfcode{get\_carbon\_data}}{\emph{start\_date}, \emph{end\_date}, \emph{groups=None}, \emph{excel\_export\_path=None}}{}
Get carbon data. Returns Pandas dataframe, optional excel export.
Returns results for all site groups if none is specified.
\begin{quote}\begin{description}
\item[{Parameters}] \leavevmode\begin{itemize}
\item {} 
\textbf{start\_date} (\emph{str}) -- start date for database query of form `01-JAN-2005'

\item {} 
\textbf{end\_date} (\emph{str}) -- end date for database query of form `01-JAN-2006'

\item {} 
\textbf{groups} (\emph{iterable of strings}) -- filter for site groups

\item {} 
\textbf{excel\_export\_path} (\emph{string or None}) -- path for MS Excel 2007 output (e.g. C:/tmp/my\_export.xlsx; default None)

\end{itemize}

\item[{Returns}] \leavevmode
query result

\item[{Return type}] \leavevmode
pandas.DataFrame

\end{description}\end{quote}

\end{fulllineitems}

\index{get\_data\_with\_alkalinity() (webb\_utils.retrieve\_data.RetrieveData method)}

\begin{fulllineitems}
\phantomsection\label{modules:webb_utils.retrieve_data.RetrieveData.get_data_with_alkalinity}\pysiglinewithargsret{\bfcode{get\_data\_with\_alkalinity}}{\emph{start\_date}, \emph{end\_date}, \emph{groups=None}, \emph{excel\_export\_path=None}}{}
Get data with alkalinity. Returns Pandas dataframe, optional excel export.
Returns results for all site groups if none is specified.
\begin{quote}\begin{description}
\item[{Parameters}] \leavevmode\begin{itemize}
\item {} 
\textbf{start\_date} (\emph{str}) -- start date for database query of form `01-JAN-2005'

\item {} 
\textbf{end\_date} (\emph{str}) -- end date for database query of form `01-JAN-2006'

\item {} 
\textbf{groups} (\emph{iterable of strings}) -- filter for site groups

\item {} 
\textbf{excel\_export\_path} (\emph{string or None}) -- path for MS Excel 2007 output (e.g. C:/tmp/my\_export.xlsx; default None)

\end{itemize}

\item[{Returns}] \leavevmode
query result

\item[{Return type}] \leavevmode
pandas.DataFrame

\end{description}\end{quote}

\end{fulllineitems}

\index{get\_piezo\_sites() (webb\_utils.retrieve\_data.RetrieveData method)}

\begin{fulllineitems}
\phantomsection\label{modules:webb_utils.retrieve_data.RetrieveData.get_piezo_sites}\pysiglinewithargsret{\bfcode{get\_piezo\_sites}}{\emph{start\_date}, \emph{end\_date}, \emph{groups=None}, \emph{excel\_export\_path=None}}{}
Get total organic carbon and total inorganic carbon.
Returns Pandas dataframe, optional excel export.
Returns results for all site groups if none is specified.
\begin{quote}\begin{description}
\item[{Parameters}] \leavevmode\begin{itemize}
\item {} 
\textbf{start\_date} (\emph{str}) -- start date for database query of form `01-JAN-2005'

\item {} 
\textbf{end\_date} (\emph{str}) -- end date for database query of form `01-JAN-2006'

\item {} 
\textbf{groups} (\emph{iterable of strings}) -- filter for site groups

\item {} 
\textbf{excel\_export\_path} (\emph{string or None}) -- path for MS Excel 2007 output (e.g. C:/tmp/my\_export.xlsx; default None)

\end{itemize}

\item[{Returns}] \leavevmode
dataframe

\item[{Return type}] \leavevmode
pandas.DataFrame

\end{description}\end{quote}

\end{fulllineitems}

\index{get\_site\_info() (webb\_utils.retrieve\_data.RetrieveData method)}

\begin{fulllineitems}
\phantomsection\label{modules:webb_utils.retrieve_data.RetrieveData.get_site_info}\pysiglinewithargsret{\bfcode{get\_site\_info}}{\emph{excel\_export\_path=None}}{}
Get site information (lat, lon, NWIS Station number, etc).
Returns Pandas dataframe, optional excel export.
\begin{quote}\begin{description}
\item[{Parameters}] \leavevmode
\textbf{excel\_export\_path} (\emph{string or None}) -- path for MS Excel 2007 output (e.g. C:/tmp/my\_export.xlsx; default None)

\item[{Returns}] \leavevmode
query result

\item[{Return type}] \leavevmode
pandas.DataFrame

\end{description}\end{quote}

\end{fulllineitems}

\index{get\_well\_check\_values() (webb\_utils.retrieve\_data.RetrieveData method)}

\begin{fulllineitems}
\phantomsection\label{modules:webb_utils.retrieve_data.RetrieveData.get_well_check_values}\pysiglinewithargsret{\bfcode{get\_well\_check\_values}}{\emph{start\_date}, \emph{end\_date}, \emph{sites=None}, \emph{excel\_export\_path=None}}{}
Get well check values. Returns Pandas dataframe, optional excel export.
Returns query results for all sites if none are specified.
\begin{quote}\begin{description}
\item[{Parameters}] \leavevmode\begin{itemize}
\item {} 
\textbf{start\_date} (\emph{str}) -- start date for database query of form `01-JAN-2005'

\item {} 
\textbf{end\_date} (\emph{str}) -- end date for database query of form `01-JAN-2006'

\item {} 
\textbf{sites} (\emph{iterable of strings}) -- filter for sites

\item {} 
\textbf{excel\_export\_path} (\emph{string or None}) -- path for MS Excel 2007 output (e.g. C:/tmp/my\_export.xlsx; default None)

\end{itemize}

\item[{Returns}] \leavevmode
query result

\item[{Return type}] \leavevmode
pandas.DataFrame

\end{description}\end{quote}

\end{fulllineitems}

\index{get\_well\_datums() (webb\_utils.retrieve\_data.RetrieveData method)}

\begin{fulllineitems}
\phantomsection\label{modules:webb_utils.retrieve_data.RetrieveData.get_well_datums}\pysiglinewithargsret{\bfcode{get\_well\_datums}}{\emph{excel\_export\_path=None}}{}
Get well datums. Returns Pandas dataframe, optional excel export.
\begin{quote}\begin{description}
\item[{Parameters}] \leavevmode
\textbf{excel\_export\_path} (\emph{string or None}) -- path for MS Excel 2007 output (e.g. C:/tmp/my\_export.xlsx; default None)

\item[{Returns}] \leavevmode
query result

\item[{Return type}] \leavevmode
pandas.DataFrame

\end{description}\end{quote}

\end{fulllineitems}

\index{get\_well\_uvs() (webb\_utils.retrieve\_data.RetrieveData method)}

\begin{fulllineitems}
\phantomsection\label{modules:webb_utils.retrieve_data.RetrieveData.get_well_uvs}\pysiglinewithargsret{\bfcode{get\_well\_uvs}}{\emph{start\_date}, \emph{end\_date}, \emph{sites=None}, \emph{excel\_export\_path=None}}{}
Get well UV data. Returns Pandas dataframe, optional excel export.
Returns query results for all sites if none are specified.
\begin{quote}\begin{description}
\item[{Parameters}] \leavevmode\begin{itemize}
\item {} 
\textbf{start\_date} (\emph{str}) -- start date for database query of form `01-JAN-2005'

\item {} 
\textbf{end\_date} (\emph{str}) -- end date for database query of form `01-JAN-2006'

\item {} 
\textbf{sites} (\emph{iterable of strings}) -- filter for sites

\item {} 
\textbf{excel\_export\_path} (\emph{string or None}) -- path for MS Excel 2007 output (e.g. C:/tmp/my\_export.xlsx; default None)

\end{itemize}

\item[{Returns}] \leavevmode
query result

\item[{Return type}] \leavevmode
pandas.DataFrame

\end{description}\end{quote}

\end{fulllineitems}


\end{fulllineitems}



\section{DB Utils}
\label{modules:db-utils}
Module containing utilites for database access.
\index{create\_db\_filter\_str() (in module webb\_utils.db\_utils)}

\begin{fulllineitems}
\phantomsection\label{modules:webb_utils.db_utils.create_db_filter_str}\pysiglinewithargsret{\code{webb\_utils.db\_utils.}\bfcode{create\_db\_filter\_str}}{\emph{param\_list}}{}
Create an Oracle safe string from a Python iterable.
\begin{quote}\begin{description}
\item[{Parameters}] \leavevmode
\textbf{param\_list} -- list of strings

\item[{Returns}] \leavevmode
Oracle safe string

\item[{Return type}] \leavevmode
string

\end{description}\end{quote}

\end{fulllineitems}

\index{AlchemDB (class in webb\_utils.db\_utils)}

\begin{fulllineitems}
\phantomsection\label{modules:webb_utils.db_utils.AlchemDB}\pysiglinewithargsret{\strong{class }\code{webb\_utils.db\_utils.}\bfcode{AlchemDB}}{\emph{schema}, \emph{password}, \emph{db\_name}}{}
Create an Oracle session.
\begin{quote}\begin{description}
\item[{Parameters}] \leavevmode\begin{itemize}
\item {} 
\textbf{schema} (\emph{str}) -- schema user name

\item {} 
\textbf{password} (\emph{str}) -- schema user password

\item {} 
\textbf{db\_name} (\emph{str}) -- database name

\end{itemize}

\end{description}\end{quote}
\index{create\_session() (webb\_utils.db\_utils.AlchemDB method)}

\begin{fulllineitems}
\phantomsection\label{modules:webb_utils.db_utils.AlchemDB.create_session}\pysiglinewithargsret{\bfcode{create\_session}}{}{}
Create an Oracle session
\begin{quote}\begin{description}
\item[{Returns}] \leavevmode
Oracle database session

\item[{Return type}] \leavevmode
sqlalchemy.orm.session.Session

\end{description}\end{quote}

\end{fulllineitems}


\end{fulllineitems}



\chapter{Example Usage}
\label{example_usage:example-usage}\label{example_usage::doc}
Illustration of using this package to retrieve data through
a Python interactive console.


\section{Data Retrieval}
\label{example_usage:data-retrieval}
This example is providing assuming that one is in the
py-webb-db-utils has been installed as a package. These
examples show MS Excel 2007 exports being placed at
C:/Users/joe/downloads/, but can be any appropriate directory.

\begin{Verbatim}[commandchars=\\\{\}]
\PYG{g+gp}{\PYGZgt{}\PYGZgt{}\PYGZgt{} }\PYG{k+kn}{from} \PYG{n+nn}{webb\PYGZus{}utils.retrieve\PYGZus{}data} \PYG{k+kn}{import} \PYG{n}{RetrieveData}
\PYG{g+gp}{\PYGZgt{}\PYGZgt{}\PYGZgt{} }\PYG{n}{SCHEMA\PYGZus{}NAME} \PYG{o}{=} \PYG{l+s}{\PYGZsq{}}\PYG{l+s}{my\PYGZus{}schema}\PYG{l+s}{\PYGZsq{}}
\PYG{g+gp}{\PYGZgt{}\PYGZgt{}\PYGZgt{} }\PYG{n}{PASSWORD} \PYG{o}{=} \PYG{l+s}{\PYGZsq{}}\PYG{l+s}{my\PYGZus{}password}\PYG{l+s}{\PYGZsq{}}
\PYG{g+gp}{\PYGZgt{}\PYGZgt{}\PYGZgt{} }\PYG{n}{DB\PYGZus{}NAME} \PYG{o}{=} \PYG{l+s}{\PYGZsq{}}\PYG{l+s}{db.name.blah}\PYG{l+s}{\PYGZsq{}}
\PYG{g+gp}{\PYGZgt{}\PYGZgt{}\PYGZgt{} }\PYG{n}{START\PYGZus{}DATE} \PYG{o}{=} \PYG{l+s}{\PYGZsq{}}\PYG{l+s}{01\PYGZhy{}OCT\PYGZhy{}2006}\PYG{l+s}{\PYGZsq{}}
\PYG{g+gp}{\PYGZgt{}\PYGZgt{}\PYGZgt{} }\PYG{n}{END\PYGZus{}DATE} \PYG{o}{=} \PYG{l+s}{\PYGZsq{}}\PYG{l+s}{30\PYGZhy{}SEP\PYGZhy{}2007}\PYG{l+s}{\PYGZsq{}}
\PYG{g+gp}{\PYGZgt{}\PYGZgt{}\PYGZgt{} }\PYG{n}{rd} \PYG{o}{=} \PYG{n}{RetrieveData}\PYG{p}{(}\PYG{n}{SCHEMA\PYGZus{}NAME}\PYG{p}{,} \PYG{n}{PASSWORD}\PYG{p}{,} \PYG{n}{DB\PYGZus{}NAME}\PYG{p}{)}
\PYG{g+gp}{\PYGZgt{}\PYGZgt{}\PYGZgt{} }\PYG{c}{\PYGZsh{} get well datums}
\PYG{g+gp}{\PYGZgt{}\PYGZgt{}\PYGZgt{} }\PYG{n}{gwd} \PYG{o}{=} \PYG{n}{rd}\PYG{o}{.}\PYG{n}{get\PYGZus{}well\PYGZus{}datums}\PYG{p}{(}\PYG{n}{excel\PYGZus{}export\PYGZus{}path}\PYG{o}{=}\PYG{l+s}{\PYGZsq{}}\PYG{l+s}{C:/Users/joe/downloads/well\PYGZus{}datums.xlsx}\PYG{l+s}{\PYGZsq{}}\PYG{p}{)}
\PYG{g+gp}{\PYGZgt{}\PYGZgt{}\PYGZgt{} }\PYG{c}{\PYGZsh{} get well uv}
\PYG{g+gp}{\PYGZgt{}\PYGZgt{}\PYGZgt{} }\PYG{n}{gwu} \PYG{o}{=} \PYG{n}{rd}\PYG{o}{.}\PYG{n}{get\PYGZus{}well\PYGZus{}uvs}\PYG{p}{(}\PYG{n}{START\PYGZus{}DATE}\PYG{p}{,} \PYG{n}{END\PYGZus{}DATE}\PYG{p}{,}
\PYG{g+go}{        excel\PYGZus{}export\PYGZus{}path=\PYGZsq{}C:/Users/joe/downloads/well\PYGZus{}uvs.xlsx\PYGZsq{})}
\PYG{g+gp}{\PYGZgt{}\PYGZgt{}\PYGZgt{} }\PYG{c}{\PYGZsh{} get carbon data}
\PYG{g+gp}{\PYGZgt{}\PYGZgt{}\PYGZgt{} }\PYG{n}{gcd} \PYG{o}{=} \PYG{n}{rd}\PYG{o}{.}\PYG{n}{get\PYGZus{}carbon\PYGZus{}data}\PYG{p}{(}\PYG{n}{START\PYGZus{}DATE}\PYG{p}{,} \PYG{n}{END\PYGZus{}DATE}\PYG{p}{,}
\PYG{g+go}{        excel\PYGZus{}export\PYGZus{}path=\PYGZsq{}C:/Users/joe/downloads/carbon\PYGZus{}data.xlsx\PYGZsq{})}
\PYG{g+gp}{\PYGZgt{}\PYGZgt{}\PYGZgt{} }\PYG{c}{\PYGZsh{} get alkalinity}
\PYG{g+gp}{\PYGZgt{}\PYGZgt{}\PYGZgt{} }\PYG{n}{gda} \PYG{o}{=} \PYG{n}{rd}\PYG{o}{.}\PYG{n}{get\PYGZus{}data\PYGZus{}with\PYGZus{}alkalinity}\PYG{p}{(}\PYG{n}{START\PYGZus{}DATE}\PYG{p}{,} \PYG{n}{END\PYGZus{}DATE}\PYG{p}{,}
\PYG{g+go}{        excel\PYGZus{}export\PYGZus{}path=\PYGZsq{}C:/Users/joe/downloads/data\PYGZus{}alk.xlsx\PYGZsq{})}
\PYG{g+gp}{\PYGZgt{}\PYGZgt{}\PYGZgt{} }\PYG{c}{\PYGZsh{} get check values}
\PYG{g+gp}{\PYGZgt{}\PYGZgt{}\PYGZgt{} }\PYG{n}{gwcv} \PYG{o}{=} \PYG{n}{rd}\PYG{o}{.}\PYG{n}{get\PYGZus{}well\PYGZus{}check\PYGZus{}values}\PYG{p}{(}\PYG{n}{START\PYGZus{}DATE}\PYG{p}{,} \PYG{n}{END\PYGZus{}DATE}\PYG{p}{,}
\PYG{g+go}{        excel\PYGZus{}export\PYGZus{}path=\PYGZsq{}C:/Users/joe/downloads/check\PYGZus{}values.xlsx\PYGZsq{})}
\PYG{g+gp}{\PYGZgt{}\PYGZgt{}\PYGZgt{} }\PYG{c}{\PYGZsh{} get piezo sites}
\PYG{g+gp}{\PYGZgt{}\PYGZgt{}\PYGZgt{} }\PYG{n}{gps} \PYG{o}{=} \PYG{n}{rd}\PYG{o}{.}\PYG{n}{get\PYGZus{}piezo\PYGZus{}sites}\PYG{p}{(}\PYG{n}{START\PYGZus{}DATE}\PYG{p}{,} \PYG{n}{END\PYGZus{}DATE}\PYG{p}{,}
\PYG{g+go}{        excel\PYGZus{}export\PYGZus{}path=\PYGZsq{}C:/Users/joe/downloads/piezo\PYGZus{}sites.xlsx\PYGZsq{})}
\PYG{g+gp}{\PYGZgt{}\PYGZgt{}\PYGZgt{} }\PYG{c}{\PYGZsh{} get site information}
\PYG{g+gp}{\PYGZgt{}\PYGZgt{}\PYGZgt{} }\PYG{n}{sites} \PYG{o}{=} \PYG{n}{rd}\PYG{o}{.}\PYG{n}{get\PYGZus{}site\PYGZus{}info}\PYG{p}{(}\PYG{n}{excel\PYGZus{}export\PYGZus{}path}\PYG{o}{=}\PYG{l+s}{\PYGZsq{}}\PYG{l+s}{C:/Users/joe/downloads/site\PYGZus{}info.xlsx}\PYG{l+s}{\PYGZsq{}}\PYG{p}{)}
\PYG{g+gp}{\PYGZgt{}\PYGZgt{}\PYGZgt{} }\PYG{c}{\PYGZsh{} close the session}
\PYG{g+gp}{\PYGZgt{}\PYGZgt{}\PYGZgt{} }\PYG{n}{rd}\PYG{o}{.}\PYG{n}{close\PYGZus{}session}\PYG{p}{(}\PYG{p}{)}
\end{Verbatim}


\chapter{Indices and tables}
\label{webb_utils_doc:indices-and-tables}\begin{itemize}
\item {} 
\emph{genindex}

\item {} 
\emph{modindex}

\item {} 
\emph{search}

\end{itemize}



\renewcommand{\indexname}{Index}
\printindex
\end{document}
