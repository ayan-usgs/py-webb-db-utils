% Generated by Sphinx.
\def\sphinxdocclass{report}
\documentclass[letterpaper,10pt,english]{sphinxmanual}
\usepackage[utf8]{inputenc}
\DeclareUnicodeCharacter{00A0}{\nobreakspace}
\usepackage{cmap}
\usepackage[T1]{fontenc}
\usepackage{babel}
\usepackage{times}
\usepackage[Bjarne]{fncychap}
\usepackage{longtable}
\usepackage{sphinx}
\usepackage{multirow}


\title{py-webb-db-utils Documentation}
\date{September 19, 2014}
\release{a1}
\author{Andrew Yan}
\newcommand{\sphinxlogo}{}
\renewcommand{\releasename}{Release}
\makeindex

\makeatletter
\def\PYG@reset{\let\PYG@it=\relax \let\PYG@bf=\relax%
    \let\PYG@ul=\relax \let\PYG@tc=\relax%
    \let\PYG@bc=\relax \let\PYG@ff=\relax}
\def\PYG@tok#1{\csname PYG@tok@#1\endcsname}
\def\PYG@toks#1+{\ifx\relax#1\empty\else%
    \PYG@tok{#1}\expandafter\PYG@toks\fi}
\def\PYG@do#1{\PYG@bc{\PYG@tc{\PYG@ul{%
    \PYG@it{\PYG@bf{\PYG@ff{#1}}}}}}}
\def\PYG#1#2{\PYG@reset\PYG@toks#1+\relax+\PYG@do{#2}}

\expandafter\def\csname PYG@tok@gd\endcsname{\def\PYG@tc##1{\textcolor[rgb]{0.63,0.00,0.00}{##1}}}
\expandafter\def\csname PYG@tok@gu\endcsname{\let\PYG@bf=\textbf\def\PYG@tc##1{\textcolor[rgb]{0.50,0.00,0.50}{##1}}}
\expandafter\def\csname PYG@tok@gt\endcsname{\def\PYG@tc##1{\textcolor[rgb]{0.00,0.27,0.87}{##1}}}
\expandafter\def\csname PYG@tok@gs\endcsname{\let\PYG@bf=\textbf}
\expandafter\def\csname PYG@tok@gr\endcsname{\def\PYG@tc##1{\textcolor[rgb]{1.00,0.00,0.00}{##1}}}
\expandafter\def\csname PYG@tok@cm\endcsname{\let\PYG@it=\textit\def\PYG@tc##1{\textcolor[rgb]{0.25,0.50,0.56}{##1}}}
\expandafter\def\csname PYG@tok@vg\endcsname{\def\PYG@tc##1{\textcolor[rgb]{0.73,0.38,0.84}{##1}}}
\expandafter\def\csname PYG@tok@m\endcsname{\def\PYG@tc##1{\textcolor[rgb]{0.13,0.50,0.31}{##1}}}
\expandafter\def\csname PYG@tok@mh\endcsname{\def\PYG@tc##1{\textcolor[rgb]{0.13,0.50,0.31}{##1}}}
\expandafter\def\csname PYG@tok@cs\endcsname{\def\PYG@tc##1{\textcolor[rgb]{0.25,0.50,0.56}{##1}}\def\PYG@bc##1{\setlength{\fboxsep}{0pt}\colorbox[rgb]{1.00,0.94,0.94}{\strut ##1}}}
\expandafter\def\csname PYG@tok@ge\endcsname{\let\PYG@it=\textit}
\expandafter\def\csname PYG@tok@vc\endcsname{\def\PYG@tc##1{\textcolor[rgb]{0.73,0.38,0.84}{##1}}}
\expandafter\def\csname PYG@tok@il\endcsname{\def\PYG@tc##1{\textcolor[rgb]{0.13,0.50,0.31}{##1}}}
\expandafter\def\csname PYG@tok@go\endcsname{\def\PYG@tc##1{\textcolor[rgb]{0.20,0.20,0.20}{##1}}}
\expandafter\def\csname PYG@tok@cp\endcsname{\def\PYG@tc##1{\textcolor[rgb]{0.00,0.44,0.13}{##1}}}
\expandafter\def\csname PYG@tok@gi\endcsname{\def\PYG@tc##1{\textcolor[rgb]{0.00,0.63,0.00}{##1}}}
\expandafter\def\csname PYG@tok@gh\endcsname{\let\PYG@bf=\textbf\def\PYG@tc##1{\textcolor[rgb]{0.00,0.00,0.50}{##1}}}
\expandafter\def\csname PYG@tok@ni\endcsname{\let\PYG@bf=\textbf\def\PYG@tc##1{\textcolor[rgb]{0.84,0.33,0.22}{##1}}}
\expandafter\def\csname PYG@tok@nl\endcsname{\let\PYG@bf=\textbf\def\PYG@tc##1{\textcolor[rgb]{0.00,0.13,0.44}{##1}}}
\expandafter\def\csname PYG@tok@nn\endcsname{\let\PYG@bf=\textbf\def\PYG@tc##1{\textcolor[rgb]{0.05,0.52,0.71}{##1}}}
\expandafter\def\csname PYG@tok@no\endcsname{\def\PYG@tc##1{\textcolor[rgb]{0.38,0.68,0.84}{##1}}}
\expandafter\def\csname PYG@tok@na\endcsname{\def\PYG@tc##1{\textcolor[rgb]{0.25,0.44,0.63}{##1}}}
\expandafter\def\csname PYG@tok@nb\endcsname{\def\PYG@tc##1{\textcolor[rgb]{0.00,0.44,0.13}{##1}}}
\expandafter\def\csname PYG@tok@nc\endcsname{\let\PYG@bf=\textbf\def\PYG@tc##1{\textcolor[rgb]{0.05,0.52,0.71}{##1}}}
\expandafter\def\csname PYG@tok@nd\endcsname{\let\PYG@bf=\textbf\def\PYG@tc##1{\textcolor[rgb]{0.33,0.33,0.33}{##1}}}
\expandafter\def\csname PYG@tok@ne\endcsname{\def\PYG@tc##1{\textcolor[rgb]{0.00,0.44,0.13}{##1}}}
\expandafter\def\csname PYG@tok@nf\endcsname{\def\PYG@tc##1{\textcolor[rgb]{0.02,0.16,0.49}{##1}}}
\expandafter\def\csname PYG@tok@si\endcsname{\let\PYG@it=\textit\def\PYG@tc##1{\textcolor[rgb]{0.44,0.63,0.82}{##1}}}
\expandafter\def\csname PYG@tok@s2\endcsname{\def\PYG@tc##1{\textcolor[rgb]{0.25,0.44,0.63}{##1}}}
\expandafter\def\csname PYG@tok@vi\endcsname{\def\PYG@tc##1{\textcolor[rgb]{0.73,0.38,0.84}{##1}}}
\expandafter\def\csname PYG@tok@nt\endcsname{\let\PYG@bf=\textbf\def\PYG@tc##1{\textcolor[rgb]{0.02,0.16,0.45}{##1}}}
\expandafter\def\csname PYG@tok@nv\endcsname{\def\PYG@tc##1{\textcolor[rgb]{0.73,0.38,0.84}{##1}}}
\expandafter\def\csname PYG@tok@s1\endcsname{\def\PYG@tc##1{\textcolor[rgb]{0.25,0.44,0.63}{##1}}}
\expandafter\def\csname PYG@tok@gp\endcsname{\let\PYG@bf=\textbf\def\PYG@tc##1{\textcolor[rgb]{0.78,0.36,0.04}{##1}}}
\expandafter\def\csname PYG@tok@sh\endcsname{\def\PYG@tc##1{\textcolor[rgb]{0.25,0.44,0.63}{##1}}}
\expandafter\def\csname PYG@tok@ow\endcsname{\let\PYG@bf=\textbf\def\PYG@tc##1{\textcolor[rgb]{0.00,0.44,0.13}{##1}}}
\expandafter\def\csname PYG@tok@sx\endcsname{\def\PYG@tc##1{\textcolor[rgb]{0.78,0.36,0.04}{##1}}}
\expandafter\def\csname PYG@tok@bp\endcsname{\def\PYG@tc##1{\textcolor[rgb]{0.00,0.44,0.13}{##1}}}
\expandafter\def\csname PYG@tok@c1\endcsname{\let\PYG@it=\textit\def\PYG@tc##1{\textcolor[rgb]{0.25,0.50,0.56}{##1}}}
\expandafter\def\csname PYG@tok@kc\endcsname{\let\PYG@bf=\textbf\def\PYG@tc##1{\textcolor[rgb]{0.00,0.44,0.13}{##1}}}
\expandafter\def\csname PYG@tok@c\endcsname{\let\PYG@it=\textit\def\PYG@tc##1{\textcolor[rgb]{0.25,0.50,0.56}{##1}}}
\expandafter\def\csname PYG@tok@mf\endcsname{\def\PYG@tc##1{\textcolor[rgb]{0.13,0.50,0.31}{##1}}}
\expandafter\def\csname PYG@tok@err\endcsname{\def\PYG@bc##1{\setlength{\fboxsep}{0pt}\fcolorbox[rgb]{1.00,0.00,0.00}{1,1,1}{\strut ##1}}}
\expandafter\def\csname PYG@tok@kd\endcsname{\let\PYG@bf=\textbf\def\PYG@tc##1{\textcolor[rgb]{0.00,0.44,0.13}{##1}}}
\expandafter\def\csname PYG@tok@ss\endcsname{\def\PYG@tc##1{\textcolor[rgb]{0.32,0.47,0.09}{##1}}}
\expandafter\def\csname PYG@tok@sr\endcsname{\def\PYG@tc##1{\textcolor[rgb]{0.14,0.33,0.53}{##1}}}
\expandafter\def\csname PYG@tok@mo\endcsname{\def\PYG@tc##1{\textcolor[rgb]{0.13,0.50,0.31}{##1}}}
\expandafter\def\csname PYG@tok@mi\endcsname{\def\PYG@tc##1{\textcolor[rgb]{0.13,0.50,0.31}{##1}}}
\expandafter\def\csname PYG@tok@kn\endcsname{\let\PYG@bf=\textbf\def\PYG@tc##1{\textcolor[rgb]{0.00,0.44,0.13}{##1}}}
\expandafter\def\csname PYG@tok@o\endcsname{\def\PYG@tc##1{\textcolor[rgb]{0.40,0.40,0.40}{##1}}}
\expandafter\def\csname PYG@tok@kr\endcsname{\let\PYG@bf=\textbf\def\PYG@tc##1{\textcolor[rgb]{0.00,0.44,0.13}{##1}}}
\expandafter\def\csname PYG@tok@s\endcsname{\def\PYG@tc##1{\textcolor[rgb]{0.25,0.44,0.63}{##1}}}
\expandafter\def\csname PYG@tok@kp\endcsname{\def\PYG@tc##1{\textcolor[rgb]{0.00,0.44,0.13}{##1}}}
\expandafter\def\csname PYG@tok@w\endcsname{\def\PYG@tc##1{\textcolor[rgb]{0.73,0.73,0.73}{##1}}}
\expandafter\def\csname PYG@tok@kt\endcsname{\def\PYG@tc##1{\textcolor[rgb]{0.56,0.13,0.00}{##1}}}
\expandafter\def\csname PYG@tok@sc\endcsname{\def\PYG@tc##1{\textcolor[rgb]{0.25,0.44,0.63}{##1}}}
\expandafter\def\csname PYG@tok@sb\endcsname{\def\PYG@tc##1{\textcolor[rgb]{0.25,0.44,0.63}{##1}}}
\expandafter\def\csname PYG@tok@k\endcsname{\let\PYG@bf=\textbf\def\PYG@tc##1{\textcolor[rgb]{0.00,0.44,0.13}{##1}}}
\expandafter\def\csname PYG@tok@se\endcsname{\let\PYG@bf=\textbf\def\PYG@tc##1{\textcolor[rgb]{0.25,0.44,0.63}{##1}}}
\expandafter\def\csname PYG@tok@sd\endcsname{\let\PYG@it=\textit\def\PYG@tc##1{\textcolor[rgb]{0.25,0.44,0.63}{##1}}}

\def\PYGZbs{\char`\\}
\def\PYGZus{\char`\_}
\def\PYGZob{\char`\{}
\def\PYGZcb{\char`\}}
\def\PYGZca{\char`\^}
\def\PYGZam{\char`\&}
\def\PYGZlt{\char`\<}
\def\PYGZgt{\char`\>}
\def\PYGZsh{\char`\#}
\def\PYGZpc{\char`\%}
\def\PYGZdl{\char`\$}
\def\PYGZhy{\char`\-}
\def\PYGZsq{\char`\'}
\def\PYGZdq{\char`\"}
\def\PYGZti{\char`\~}
% for compatibility with earlier versions
\def\PYGZat{@}
\def\PYGZlb{[}
\def\PYGZrb{]}
\makeatother

\renewcommand\PYGZsq{\textquotesingle}

\begin{document}

\maketitle
\tableofcontents
\phantomsection\label{webb_utils_doc::doc}


Contents:


\chapter{Package Installation}
\label{installation::doc}\label{installation:welcome-to-py-webb-db-utils-s-documentation}\label{installation:package-installation}
If not already installed, you will want to install Python 2.7. Downloads can be found
at \href{https://www.python.org/download/releases/}{https://www.python.org/download/releases/}. The 32-bit verson of Python is recommended
as it is better supported by the Python community.

A C complier may also need to be installed and setup. This may take of form of GCC,
MS Visual Studio 2008, or MinGW32.

There are a number of Python package dependencies that are required for the running
of this package. These dependencies are summarized in requirements.txt and are summarized
here for convenience. All of these packages may be installed using pip from the command line
(e.g. \code{pip install python-dateutil}).
\begin{itemize}
\item {} 
SQLAlchemy==0.9.7

\item {} 
cx-Oracle==5.1.3

\item {} 
numpy==1.8.1

\item {} 
openpyxl==1.8.6

\item {} 
pandas==0.14.1

\item {} 
python-dateutil==2.2

\item {} 
pytz==2014.4

\item {} 
six==1.7.3

\end{itemize}

Some of these dependencies may prove difficult to install on Windows. If this is a problem, unofficial
binaries of these packages may be found from the University of California, Irvine at
\href{http://www.lfd.uci.edu/~gohlke/pythonlibs/}{http://www.lfd.uci.edu/\textasciitilde{}gohlke/pythonlibs/}.

Once dependenacies are installed, the package may be installed can be installed through the following steps:
\begin{itemize}
\item {} 
In the command line, change to the py-webb-db-utils directory

\item {} 
Run the following command: \code{python setup.py install}

\end{itemize}


\chapter{Description of Package Modules}
\label{modules::doc}\label{modules:description-of-package-modules}

\section{Base SQL}
\label{modules:base-sql}
Module containing the SQL used in the database queries.
Should the SQL need to be changed, this is the place to
do it.


\section{Sites}
\label{modules:sites}
List of sites and site groups. This sites may be changed
if a new site is added to the WEBB project.


\section{Retrieve Data}
\label{modules:retrieve-data}
Module containing classes for creating objects to
effect data retrieval from an Oracle database.
\index{RetrieveData (class in webb\_utils.retrieve\_data)}

\begin{fulllineitems}
\phantomsection\label{modules:webb_utils.retrieve_data.RetrieveData}\pysiglinewithargsret{\strong{class }\code{webb\_utils.retrieve\_data.}\bfcode{RetrieveData}}{\emph{schema}, \emph{password}, \emph{db\_name}, \emph{excel\_indexes=False}}{}
Retrieve data object to execute and retrieve the results
of six database queries deemed important to the USGS WEBB
project.
\begin{quote}\begin{description}
\item[{Parameters}] \leavevmode\begin{itemize}
\item {} 
\textbf{schema} (\emph{str}) -- schema user name

\item {} 
\textbf{password} (\emph{str}) -- schema user password

\item {} 
\textbf{db\_name} (\emph{str}) -- database name

\item {} 
\textbf{excel\_indexes} (\emph{bool}) -- whether an exported excel file should have Pandas dataframe indexes; default is False

\end{itemize}

\end{description}\end{quote}
\index{\_create\_dataframe() (webb\_utils.retrieve\_data.RetrieveData method)}

\begin{fulllineitems}
\phantomsection\label{modules:webb_utils.retrieve_data.RetrieveData._create_dataframe}\pysiglinewithargsret{\bfcode{\_create\_dataframe}}{\emph{data}, \emph{columns}}{}
Internal method to create a pandas dataframe.
\begin{quote}\begin{description}
\item[{Parameters}] \leavevmode\begin{itemize}
\item {} 
\textbf{data} (\emph{list of tuples}) -- raw query results

\item {} 
\textbf{columns} (\emph{list of strings}) -- dataframe column names

\end{itemize}

\item[{Returns}] \leavevmode
dataframe of query results

\item[{Return type}] \leavevmode
pandas.DataFrame

\end{description}\end{quote}

\end{fulllineitems}

\index{get\_carbon\_data() (webb\_utils.retrieve\_data.RetrieveData method)}

\begin{fulllineitems}
\phantomsection\label{modules:webb_utils.retrieve_data.RetrieveData.get_carbon_data}\pysiglinewithargsret{\bfcode{get\_carbon\_data}}{\emph{start\_date}, \emph{end\_date}, \emph{groups=None}, \emph{excel\_export\_path=None}}{}
Get carbon data. Returns Pandas dataframe, optional excel export.
Returns results for all site groups if none is specified.
\begin{quote}\begin{description}
\item[{Parameters}] \leavevmode\begin{itemize}
\item {} 
\textbf{start\_date} (\emph{str}) -- start date for database query of form `01-JAN-2005'

\item {} 
\textbf{end\_date} (\emph{str}) -- end date for database query of form `01-JAN-2006'

\item {} 
\textbf{groups} (\emph{iterable of strings}) -- filter for site groups

\item {} 
\textbf{excel\_export\_path} (\emph{string or None}) -- path for MS Excel 2007 output (e.g. C:/tmp/my\_export.xlsx; default None)

\end{itemize}

\item[{Returns}] \leavevmode
query result

\item[{Return type}] \leavevmode
pandas.DataFrame

\end{description}\end{quote}

\end{fulllineitems}

\index{get\_data\_with\_alkalinity() (webb\_utils.retrieve\_data.RetrieveData method)}

\begin{fulllineitems}
\phantomsection\label{modules:webb_utils.retrieve_data.RetrieveData.get_data_with_alkalinity}\pysiglinewithargsret{\bfcode{get\_data\_with\_alkalinity}}{\emph{start\_date}, \emph{end\_date}, \emph{groups=None}, \emph{excel\_export\_path=None}}{}
Get data with alkalinity. Returns Pandas dataframe, optional excel export.
Returns results for all site groups if none is specified.
\begin{quote}\begin{description}
\item[{Parameters}] \leavevmode\begin{itemize}
\item {} 
\textbf{start\_date} (\emph{str}) -- start date for database query of form `01-JAN-2005'

\item {} 
\textbf{end\_date} (\emph{str}) -- end date for database query of form `01-JAN-2006'

\item {} 
\textbf{groups} (\emph{iterable of strings}) -- filter for site groups

\item {} 
\textbf{excel\_export\_path} (\emph{string or None}) -- path for MS Excel 2007 output (e.g. C:/tmp/my\_export.xlsx; default None)

\end{itemize}

\item[{Returns}] \leavevmode
query result

\item[{Return type}] \leavevmode
pandas.DataFrame

\end{description}\end{quote}

\end{fulllineitems}

\index{get\_piezo\_sites() (webb\_utils.retrieve\_data.RetrieveData method)}

\begin{fulllineitems}
\phantomsection\label{modules:webb_utils.retrieve_data.RetrieveData.get_piezo_sites}\pysiglinewithargsret{\bfcode{get\_piezo\_sites}}{\emph{start\_date}, \emph{end\_date}, \emph{groups=None}, \emph{excel\_export\_path=None}}{}
Get total organic carbon and total inorganic carbon.
Returns Pandas dataframe, optional excel export.
Returns results for all site groups if none is specified.
\begin{quote}\begin{description}
\item[{Parameters}] \leavevmode\begin{itemize}
\item {} 
\textbf{start\_date} (\emph{str}) -- start date for database query of form `01-JAN-2005'

\item {} 
\textbf{end\_date} (\emph{str}) -- end date for database query of form `01-JAN-2006'

\item {} 
\textbf{groups} (\emph{iterable of strings}) -- filter for site groups

\item {} 
\textbf{excel\_export\_path} (\emph{string or None}) -- path for MS Excel 2007 output (e.g. C:/tmp/my\_export.xlsx; default None)

\end{itemize}

\item[{Returns}] \leavevmode
dataframe

\item[{Return type}] \leavevmode
pandas.DataFrame

\end{description}\end{quote}

\end{fulllineitems}

\index{get\_site\_info() (webb\_utils.retrieve\_data.RetrieveData method)}

\begin{fulllineitems}
\phantomsection\label{modules:webb_utils.retrieve_data.RetrieveData.get_site_info}\pysiglinewithargsret{\bfcode{get\_site\_info}}{\emph{excel\_export\_path=None}}{}
Get site information (lat, lon, NWIS Station number, etc).
Returns Pandas dataframe, optional excel export.
\begin{quote}\begin{description}
\item[{Parameters}] \leavevmode
\textbf{excel\_export\_path} (\emph{string or None}) -- path for MS Excel 2007 output (e.g. C:/tmp/my\_export.xlsx; default None)

\item[{Returns}] \leavevmode
query result

\item[{Return type}] \leavevmode
pandas.DataFrame

\end{description}\end{quote}

\end{fulllineitems}

\index{get\_well\_check\_values() (webb\_utils.retrieve\_data.RetrieveData method)}

\begin{fulllineitems}
\phantomsection\label{modules:webb_utils.retrieve_data.RetrieveData.get_well_check_values}\pysiglinewithargsret{\bfcode{get\_well\_check\_values}}{\emph{start\_date}, \emph{end\_date}, \emph{sites=None}, \emph{excel\_export\_path=None}}{}
Get well check values. Returns Pandas dataframe, optional excel export.
Returns query results for all sites if none are specified.
\begin{quote}\begin{description}
\item[{Parameters}] \leavevmode\begin{itemize}
\item {} 
\textbf{start\_date} (\emph{str}) -- start date for database query of form `01-JAN-2005'

\item {} 
\textbf{end\_date} (\emph{str}) -- end date for database query of form `01-JAN-2006'

\item {} 
\textbf{sites} (\emph{iterable of strings}) -- filter for sites

\item {} 
\textbf{excel\_export\_path} (\emph{string or None}) -- path for MS Excel 2007 output (e.g. C:/tmp/my\_export.xlsx; default None)

\end{itemize}

\item[{Returns}] \leavevmode
query result

\item[{Return type}] \leavevmode
pandas.DataFrame

\end{description}\end{quote}

\end{fulllineitems}

\index{get\_well\_datums() (webb\_utils.retrieve\_data.RetrieveData method)}

\begin{fulllineitems}
\phantomsection\label{modules:webb_utils.retrieve_data.RetrieveData.get_well_datums}\pysiglinewithargsret{\bfcode{get\_well\_datums}}{\emph{excel\_export\_path=None}}{}
Get well datums. Returns Pandas dataframe, optional excel export.
\begin{quote}\begin{description}
\item[{Parameters}] \leavevmode
\textbf{excel\_export\_path} (\emph{string or None}) -- path for MS Excel 2007 output (e.g. C:/tmp/my\_export.xlsx; default None)

\item[{Returns}] \leavevmode
query result

\item[{Return type}] \leavevmode
pandas.DataFrame

\end{description}\end{quote}

\end{fulllineitems}

\index{get\_well\_uvs() (webb\_utils.retrieve\_data.RetrieveData method)}

\begin{fulllineitems}
\phantomsection\label{modules:webb_utils.retrieve_data.RetrieveData.get_well_uvs}\pysiglinewithargsret{\bfcode{get\_well\_uvs}}{\emph{start\_date}, \emph{end\_date}, \emph{sites=None}, \emph{excel\_export\_path=None}}{}
Get well UV data. Returns Pandas dataframe, optional excel export.
Returns query results for all sites if none are specified.
\begin{quote}\begin{description}
\item[{Parameters}] \leavevmode\begin{itemize}
\item {} 
\textbf{start\_date} (\emph{str}) -- start date for database query of form `01-JAN-2005'

\item {} 
\textbf{end\_date} (\emph{str}) -- end date for database query of form `01-JAN-2006'

\item {} 
\textbf{sites} (\emph{iterable of strings}) -- filter for sites

\item {} 
\textbf{excel\_export\_path} (\emph{string or None}) -- path for MS Excel 2007 output (e.g. C:/tmp/my\_export.xlsx; default None)

\end{itemize}

\item[{Returns}] \leavevmode
query result

\item[{Return type}] \leavevmode
pandas.DataFrame

\end{description}\end{quote}

\end{fulllineitems}


\end{fulllineitems}



\section{Upload Data}
\label{modules:upload-data}
Module functions and classes to effect the upload
of new data into the data from CSV files provided
by a laboratory.
\index{string\_to\_datetime() (in module webb\_utils.upload\_data)}

\begin{fulllineitems}
\phantomsection\label{modules:webb_utils.upload_data.string_to_datetime}\pysiglinewithargsret{\code{webb\_utils.upload\_data.}\bfcode{string\_to\_datetime}}{\emph{series}, \emph{date\_col}, \emph{time\_col}, \emph{datetime\_format='\%m/\%d/\%y \%H:\%M:\%S'}}{}
Convert a string of the format specified in the datetime\_format
parameter to a Python datatime object.
\begin{quote}\begin{description}
\item[{Parameters}] \leavevmode\begin{itemize}
\item {} 
\textbf{series} (\emph{pandas.Series}) -- a pandas series within a dataframe

\item {} 
\textbf{date\_col} (\emph{int}) -- column index with dates (start counting from the left of the csv starting with 0)

\item {} 
\textbf{time\_col} (\emph{int}) -- column index with time (start counting from the left of the csv starting with 0)

\item {} 
\textbf{datetime\_format} (\emph{str}) -- string specifying the datetime format as Python directives

\end{itemize}

\item[{Returns}] \leavevmode
date

\item[{Return type}] \leavevmode
datetime.datetime

\end{description}\end{quote}

\end{fulllineitems}

\index{str\_to\_date() (in module webb\_utils.upload\_data)}

\begin{fulllineitems}
\phantomsection\label{modules:webb_utils.upload_data.str_to_date}\pysiglinewithargsret{\code{webb\_utils.upload\_data.}\bfcode{str\_to\_date}}{\emph{date\_str}, \emph{date\_format='\%m/\%d/\%y'}}{}
Convert a date string to a Python date object.
\begin{quote}\begin{description}
\item[{Parameters}] \leavevmode\begin{itemize}
\item {} 
\textbf{date\_str} (\emph{str}) -- date string

\item {} 
\textbf{date\_format} (\emph{str}) -- string specifying the date format as Python directives

\end{itemize}

\item[{Returns}] \leavevmode
date

\item[{Return type}] \leavevmode
datetime.datetime.date

\end{description}\end{quote}

\end{fulllineitems}

\index{clean\_string\_elements() (in module webb\_utils.upload\_data)}

\begin{fulllineitems}
\phantomsection\label{modules:webb_utils.upload_data.clean_string_elements}\pysiglinewithargsret{\code{webb\_utils.upload\_data.}\bfcode{clean\_string\_elements}}{\emph{element}}{}
Remove unsafe characters from strings.
\begin{quote}\begin{description}
\item[{Parameters}] \leavevmode
\textbf{element} (\emph{pandas.DataFrame or pandas.Series element}) -- a piece of data

\item[{Returns}] \leavevmode
UTF-8 safe value

\item[{Return type}] \leavevmode
pandas.DataFrame or pandas.Series element

\end{description}\end{quote}

\end{fulllineitems}

\index{UploadData (class in webb\_utils.upload\_data)}

\begin{fulllineitems}
\phantomsection\label{modules:webb_utils.upload_data.UploadData}\pysiglinewithargsret{\strong{class }\code{webb\_utils.upload\_data.}\bfcode{UploadData}}{\emph{schema}, \emph{password}, \emph{db\_name}, \emph{commit=True}}{}
Upload data from CSV files into the WEBB database.
CSV files are expected to be tab delimited by default.

All columns in the CSV file must be in the same order as the
column order specified in the db\_mappings.upload\_columns
module. In addition, the CSV file itself must not have 
a header column. By default, dates are expected to be of
from mm/dd/yy (e.g `02/23/14') and times are expected to
be of form HH:MM:SS (e.g. `15:01:45') within the CSV
file.
\begin{quote}\begin{description}
\item[{Parameters}] \leavevmode\begin{itemize}
\item {} 
\textbf{schema} (\emph{str}) -- schema user name

\item {} 
\textbf{password} (\emph{str}) -- schema user password

\item {} 
\textbf{db\_name} (\emph{str}) -- database name

\item {} 
\textbf{commit} (\emph{bool}) -- determines whether to commit after each laod method is run (defaults to True)

\end{itemize}

\end{description}\end{quote}
\index{close\_session() (webb\_utils.upload\_data.UploadData method)}

\begin{fulllineitems}
\phantomsection\label{modules:webb_utils.upload_data.UploadData.close_session}\pysiglinewithargsret{\bfcode{close\_session}}{}{}
Close the current database session.

\end{fulllineitems}

\index{commit\_session\_loads() (webb\_utils.upload\_data.UploadData method)}

\begin{fulllineitems}
\phantomsection\label{modules:webb_utils.upload_data.UploadData.commit_session_loads}\pysiglinewithargsret{\bfcode{commit\_session\_loads}}{}{}
Allows manual commit of data loaded
into the database.

\end{fulllineitems}

\index{load\_anion\_data() (webb\_utils.upload\_data.UploadData method)}

\begin{fulllineitems}
\phantomsection\label{modules:webb_utils.upload_data.UploadData.load_anion_data}\pysiglinewithargsret{\bfcode{load\_anion\_data}}{\emph{csv\_pathname}}{}
Load anion data from a csv file into the database.
\begin{quote}\begin{description}
\item[{Parameters}] \leavevmode
\textbf{csv\_pathname} (\emph{str}) -- path to the csv file

\item[{Returns}] \leavevmode
message detailing number of records loaded.

\item[{Return type}] \leavevmode
str

\end{description}\end{quote}

\end{fulllineitems}

\index{load\_bullen\_cation\_data() (webb\_utils.upload\_data.UploadData method)}

\begin{fulllineitems}
\phantomsection\label{modules:webb_utils.upload_data.UploadData.load_bullen_cation_data}\pysiglinewithargsret{\bfcode{load\_bullen\_cation\_data}}{\emph{csv\_pathname}}{}
Load cation data analyzed by the Menlo Park Strontium
Isotope Lab from a csv file into the database.
\begin{quote}\begin{description}
\item[{Parameters}] \leavevmode
\textbf{csv\_pathname} (\emph{str}) -- path to the csv file

\item[{Returns}] \leavevmode
message detailing number of records loaded.

\item[{Return type}] \leavevmode
str

\end{description}\end{quote}

\end{fulllineitems}

\index{load\_carbon\_data() (webb\_utils.upload\_data.UploadData method)}

\begin{fulllineitems}
\phantomsection\label{modules:webb_utils.upload_data.UploadData.load_carbon_data}\pysiglinewithargsret{\bfcode{load\_carbon\_data}}{\emph{csv\_pathname}}{}
Load carbon data from a csv file into the database.
\begin{quote}\begin{description}
\item[{Parameters}] \leavevmode
\textbf{csv\_pathname} (\emph{str}) -- path to the csv file

\item[{Returns}] \leavevmode
message detailing number of records loaded.

\item[{Return type}] \leavevmode
str

\end{description}\end{quote}

\end{fulllineitems}

\index{load\_carbon\_gas\_data() (webb\_utils.upload\_data.UploadData method)}

\begin{fulllineitems}
\phantomsection\label{modules:webb_utils.upload_data.UploadData.load_carbon_gas_data}\pysiglinewithargsret{\bfcode{load\_carbon\_gas\_data}}{\emph{csv\_pathname}}{}
Load carbon gas data from a csv file into the database.
\begin{quote}\begin{description}
\item[{Parameters}] \leavevmode
\textbf{csv\_pathname} (\emph{str}) -- path to the csv file

\item[{Returns}] \leavevmode
message detailing number of records loaded.

\item[{Return type}] \leavevmode
str

\end{description}\end{quote}

\end{fulllineitems}

\index{load\_cation\_data() (webb\_utils.upload\_data.UploadData method)}

\begin{fulllineitems}
\phantomsection\label{modules:webb_utils.upload_data.UploadData.load_cation_data}\pysiglinewithargsret{\bfcode{load\_cation\_data}}{\emph{csv\_pathname}}{}
Load cation data from a csv file into the database.
\begin{quote}\begin{description}
\item[{Parameters}] \leavevmode
\textbf{csv\_pathname} (\emph{str}) -- path to the csv file

\item[{Returns}] \leavevmode
message detailing number of records loaded.

\item[{Return type}] \leavevmode
str

\end{description}\end{quote}

\end{fulllineitems}

\index{load\_dv\_results\_data() (webb\_utils.upload\_data.UploadData method)}

\begin{fulllineitems}
\phantomsection\label{modules:webb_utils.upload_data.UploadData.load_dv_results_data}\pysiglinewithargsret{\bfcode{load\_dv\_results\_data}}{\emph{csv\_pathname}}{}
Load DV result data from a csv file into the database.
\begin{quote}\begin{description}
\item[{Parameters}] \leavevmode
\textbf{csv\_pathname} (\emph{str}) -- path to the csv file

\item[{Returns}] \leavevmode
message detailing number of records loaded.

\item[{Return type}] \leavevmode
str

\end{description}\end{quote}

\end{fulllineitems}

\index{load\_field\_data() (webb\_utils.upload\_data.UploadData method)}

\begin{fulllineitems}
\phantomsection\label{modules:webb_utils.upload_data.UploadData.load_field_data}\pysiglinewithargsret{\bfcode{load\_field\_data}}{\emph{csv\_pathname}}{}
Load field data from a csv file into the database.
\begin{quote}\begin{description}
\item[{Parameters}] \leavevmode
\textbf{csv\_pathname} (\emph{str}) -- path to the csv file

\item[{Returns}] \leavevmode
message detailing number of records loaded.

\item[{Return type}] \leavevmode
str

\end{description}\end{quote}

\end{fulllineitems}

\index{load\_flux\_chamber\_data() (webb\_utils.upload\_data.UploadData method)}

\begin{fulllineitems}
\phantomsection\label{modules:webb_utils.upload_data.UploadData.load_flux_chamber_data}\pysiglinewithargsret{\bfcode{load\_flux\_chamber\_data}}{\emph{csv\_pathname}}{}
Load flux chamber data from a csv file into the database.
\begin{quote}\begin{description}
\item[{Parameters}] \leavevmode
\textbf{csv\_pathname} (\emph{str}) -- path to the csv file

\item[{Returns}] \leavevmode
message detailing number of records loaded.

\item[{Return type}] \leavevmode
str

\end{description}\end{quote}

\end{fulllineitems}

\index{load\_gage\_ht\_meas\_data() (webb\_utils.upload\_data.UploadData method)}

\begin{fulllineitems}
\phantomsection\label{modules:webb_utils.upload_data.UploadData.load_gage_ht_meas_data}\pysiglinewithargsret{\bfcode{load\_gage\_ht\_meas\_data}}{\emph{csv\_pathname}, \emph{date\_col=1}, \emph{time\_col=2}}{}
Load anion data from a csv file into the database.
\begin{quote}\begin{description}
\item[{Parameters}] \leavevmode\begin{itemize}
\item {} 
\textbf{csv\_pathname} (\emph{str}) -- path to the csv file

\item {} 
\textbf{date\_col} (\emph{int}) -- column index with dates (start counting from the left of the csv starting with 0; defaults to 1)

\item {} 
\textbf{time\_col} (\emph{int}) -- column index with time (start counting from the left of the csv starting with 0; defaults to 2)

\end{itemize}

\item[{Returns}] \leavevmode
message detailing number of records loaded.

\item[{Return type}] \leavevmode
str

\end{description}\end{quote}

\end{fulllineitems}

\index{load\_gage\_ht\_rp\_data() (webb\_utils.upload\_data.UploadData method)}

\begin{fulllineitems}
\phantomsection\label{modules:webb_utils.upload_data.UploadData.load_gage_ht_rp_data}\pysiglinewithargsret{\bfcode{load\_gage\_ht\_rp\_data}}{\emph{csv\_pathname}}{}
Load gage height rp data from a csv file into the database.
\begin{quote}\begin{description}
\item[{Parameters}] \leavevmode
\textbf{csv\_pathname} (\emph{str}) -- path to the csv file

\item[{Returns}] \leavevmode
message detailing number of records loaded.

\item[{Return type}] \leavevmode
str

\end{description}\end{quote}

\end{fulllineitems}

\index{load\_mercury\_data() (webb\_utils.upload\_data.UploadData method)}

\begin{fulllineitems}
\phantomsection\label{modules:webb_utils.upload_data.UploadData.load_mercury_data}\pysiglinewithargsret{\bfcode{load\_mercury\_data}}{\emph{csv\_pathname}}{}
Load mercury data from a csv file into the database.
\begin{quote}\begin{description}
\item[{Parameters}] \leavevmode
\textbf{csv\_pathname} (\emph{str}) -- path to the csv file

\item[{Returns}] \leavevmode
message detailing number of records loaded.

\item[{Return type}] \leavevmode
str

\end{description}\end{quote}

\end{fulllineitems}

\index{load\_nutrient\_data() (webb\_utils.upload\_data.UploadData method)}

\begin{fulllineitems}
\phantomsection\label{modules:webb_utils.upload_data.UploadData.load_nutrient_data}\pysiglinewithargsret{\bfcode{load\_nutrient\_data}}{\emph{csv\_pathname}}{}
Load nutrient data from a csv file into the database.
\begin{quote}\begin{description}
\item[{Parameters}] \leavevmode
\textbf{csv\_pathname} (\emph{str}) -- path to the csv file

\item[{Returns}] \leavevmode
message detailing number of records loaded.

\item[{Return type}] \leavevmode
str

\end{description}\end{quote}

\end{fulllineitems}

\index{load\_parameters() (webb\_utils.upload\_data.UploadData method)}

\begin{fulllineitems}
\phantomsection\label{modules:webb_utils.upload_data.UploadData.load_parameters}\pysiglinewithargsret{\bfcode{load\_parameters}}{\emph{csv\_pathname}}{}
Load parameters data from a csv file into the database.
\begin{quote}\begin{description}
\item[{Parameters}] \leavevmode
\textbf{csv\_pathname} (\emph{str}) -- path to the csv file

\item[{Returns}] \leavevmode
message detailing number of records loaded.

\item[{Return type}] \leavevmode
str

\end{description}\end{quote}

\end{fulllineitems}

\index{load\_qmeas\_data() (webb\_utils.upload\_data.UploadData method)}

\begin{fulllineitems}
\phantomsection\label{modules:webb_utils.upload_data.UploadData.load_qmeas_data}\pysiglinewithargsret{\bfcode{load\_qmeas\_data}}{\emph{csv\_pathname}}{}
Load qmeas data from a csv file into the database.
\begin{quote}\begin{description}
\item[{Parameters}] \leavevmode
\textbf{csv\_pathname} (\emph{str}) -- path to the csv file

\item[{Returns}] \leavevmode
message detailing number of records loaded.

\item[{Return type}] \leavevmode
str

\end{description}\end{quote}

\end{fulllineitems}

\index{load\_rare\_cation\_data() (webb\_utils.upload\_data.UploadData method)}

\begin{fulllineitems}
\phantomsection\label{modules:webb_utils.upload_data.UploadData.load_rare_cation_data}\pysiglinewithargsret{\bfcode{load\_rare\_cation\_data}}{\emph{csv\_pathname}}{}
Load rare cation data from a csv file into the database.
\begin{quote}\begin{description}
\item[{Parameters}] \leavevmode
\textbf{csv\_pathname} (\emph{str}) -- path to the csv file

\item[{Returns}] \leavevmode
message detailing number of records loaded.

\item[{Return type}] \leavevmode
str

\end{description}\end{quote}

\end{fulllineitems}

\index{load\_raw\_cation\_data() (webb\_utils.upload\_data.UploadData method)}

\begin{fulllineitems}
\phantomsection\label{modules:webb_utils.upload_data.UploadData.load_raw_cation_data}\pysiglinewithargsret{\bfcode{load\_raw\_cation\_data}}{\emph{csv\_pathname}}{}
Load raw cation data from a csv file into the database.
\begin{quote}\begin{description}
\item[{Parameters}] \leavevmode
\textbf{csv\_pathname} (\emph{str}) -- path to the csv file

\item[{Returns}] \leavevmode
message detailing number of records loaded.

\item[{Return type}] \leavevmode
str

\end{description}\end{quote}

\end{fulllineitems}

\index{load\_rp\_desc\_data() (webb\_utils.upload\_data.UploadData method)}

\begin{fulllineitems}
\phantomsection\label{modules:webb_utils.upload_data.UploadData.load_rp_desc_data}\pysiglinewithargsret{\bfcode{load\_rp\_desc\_data}}{\emph{csv\_pathname}}{}
Load rp data from a csv file into the database.
\begin{quote}\begin{description}
\item[{Parameters}] \leavevmode
\textbf{csv\_pathname} (\emph{str}) -- path to the csv file

\item[{Returns}] \leavevmode
message detailing number of records loaded.

\item[{Return type}] \leavevmode
str

\end{description}\end{quote}

\end{fulllineitems}

\index{load\_sample\_data() (webb\_utils.upload\_data.UploadData method)}

\begin{fulllineitems}
\phantomsection\label{modules:webb_utils.upload_data.UploadData.load_sample_data}\pysiglinewithargsret{\bfcode{load\_sample\_data}}{\emph{csv\_pathname}, \emph{date\_col=2}, \emph{time\_col=3}}{}
Load sample data from a csv file into the database.
\begin{quote}\begin{description}
\item[{Parameters}] \leavevmode\begin{itemize}
\item {} 
\textbf{csv\_pathname} (\emph{str}) -- path to the csv file

\item {} 
\textbf{date\_col} (\emph{int}) -- column index with dates (start counting from the left of the csv starting with 0; defaults to 1)

\item {} 
\textbf{time\_col} (\emph{int}) -- column index with time (start counting from the left of the csv starting with 0; defaults to 2)

\end{itemize}

\item[{Returns}] \leavevmode
message detailing number of records loaded.

\item[{Return type}] \leavevmode
str

\end{description}\end{quote}

\end{fulllineitems}

\index{load\_sample\_group\_data() (webb\_utils.upload\_data.UploadData method)}

\begin{fulllineitems}
\phantomsection\label{modules:webb_utils.upload_data.UploadData.load_sample_group_data}\pysiglinewithargsret{\bfcode{load\_sample\_group\_data}}{\emph{csv\_pathname}}{}
Load sample group data from a csv file into the database.
\begin{quote}\begin{description}
\item[{Parameters}] \leavevmode
\textbf{csv\_pathname} (\emph{str}) -- path to the csv file

\item[{Returns}] \leavevmode
message detailing number of records loaded.

\item[{Return type}] \leavevmode
str

\end{description}\end{quote}

\end{fulllineitems}

\index{load\_site\_data() (webb\_utils.upload\_data.UploadData method)}

\begin{fulllineitems}
\phantomsection\label{modules:webb_utils.upload_data.UploadData.load_site_data}\pysiglinewithargsret{\bfcode{load\_site\_data}}{\emph{csv\_pathname}}{}
Load new sites from a csv file into the database.
\begin{quote}\begin{description}
\item[{Parameters}] \leavevmode
\textbf{csv\_pathname} (\emph{str}) -- path to the csv file

\item[{Returns}] \leavevmode
message detailing number of records loaded.

\item[{Return type}] \leavevmode
str

\end{description}\end{quote}

\end{fulllineitems}

\index{load\_soil\_profile\_data() (webb\_utils.upload\_data.UploadData method)}

\begin{fulllineitems}
\phantomsection\label{modules:webb_utils.upload_data.UploadData.load_soil_profile_data}\pysiglinewithargsret{\bfcode{load\_soil\_profile\_data}}{\emph{csv\_pathname}, \emph{date\_col=2}, \emph{time\_col=3}}{}
Load soil profile data from a csv file into the database.
\begin{quote}\begin{description}
\item[{Parameters}] \leavevmode\begin{itemize}
\item {} 
\textbf{csv\_pathname} (\emph{str}) -- path to the csv file

\item {} 
\textbf{date\_col} (\emph{int}) -- column index with dates (start counting from the left of the csv starting with 0; defaults to 1)

\item {} 
\textbf{time\_col} (\emph{int}) -- column index with time (start counting from the left of the csv starting with 0; defaults to 2)

\end{itemize}

\item[{Returns}] \leavevmode
message detailing number of records loaded.

\item[{Return type}] \leavevmode
str

\end{description}\end{quote}

\end{fulllineitems}

\index{load\_strontium\_isotope\_data() (webb\_utils.upload\_data.UploadData method)}

\begin{fulllineitems}
\phantomsection\label{modules:webb_utils.upload_data.UploadData.load_strontium_isotope_data}\pysiglinewithargsret{\bfcode{load\_strontium\_isotope\_data}}{\emph{csv\_pathname}}{}
Load strontium data from a csv file into the database.
\begin{quote}\begin{description}
\item[{Parameters}] \leavevmode
\textbf{csv\_pathname} (\emph{str}) -- path to the csv file

\item[{Returns}] \leavevmode
message detailing number of records loaded.

\item[{Return type}] \leavevmode
str

\end{description}\end{quote}

\end{fulllineitems}

\index{load\_test\_site\_data() (webb\_utils.upload\_data.UploadData method)}

\begin{fulllineitems}
\phantomsection\label{modules:webb_utils.upload_data.UploadData.load_test_site_data}\pysiglinewithargsret{\bfcode{load\_test\_site\_data}}{\emph{csv\_pathname}}{}
Load test site data from a csv file into the database.
\begin{quote}\begin{description}
\item[{Parameters}] \leavevmode
\textbf{csv\_pathname} (\emph{str}) -- path to the csv file

\item[{Returns}] \leavevmode
message detailing number of records loaded.

\item[{Return type}] \leavevmode
str

\end{description}\end{quote}

\end{fulllineitems}

\index{load\_uv\_results\_data() (webb\_utils.upload\_data.UploadData method)}

\begin{fulllineitems}
\phantomsection\label{modules:webb_utils.upload_data.UploadData.load_uv_results_data}\pysiglinewithargsret{\bfcode{load\_uv\_results\_data}}{\emph{csv\_pathname}, \emph{date\_col=2}, \emph{time\_col=3}}{}
Load UV result data from a csv file into the database.
\begin{quote}\begin{description}
\item[{Parameters}] \leavevmode\begin{itemize}
\item {} 
\textbf{csv\_pathname} (\emph{str}) -- path to the csv file

\item {} 
\textbf{date\_col} (\emph{int}) -- column index with dates (start counting from the left of the csv starting with 0; defaults to 2)

\item {} 
\textbf{time\_col} (\emph{int}) -- column index with time (start counting from the left of the csv starting with 0; defaults to 3)

\end{itemize}

\item[{Returns}] \leavevmode
message detailing number of records loaded.

\item[{Return type}] \leavevmode
str

\end{description}\end{quote}

\end{fulllineitems}

\index{load\_water\_isotope\_data() (webb\_utils.upload\_data.UploadData method)}

\begin{fulllineitems}
\phantomsection\label{modules:webb_utils.upload_data.UploadData.load_water_isotope_data}\pysiglinewithargsret{\bfcode{load\_water\_isotope\_data}}{\emph{csv\_pathname}}{}
Load water isotope data from a csv file into the database.
\begin{quote}\begin{description}
\item[{Parameters}] \leavevmode
\textbf{csv\_pathname} (\emph{str}) -- path to the csv file

\item[{Returns}] \leavevmode
message detailing number of records loaded.

\item[{Return type}] \leavevmode
str

\end{description}\end{quote}

\end{fulllineitems}

\index{load\_wellhead\_measurement\_data() (webb\_utils.upload\_data.UploadData method)}

\begin{fulllineitems}
\phantomsection\label{modules:webb_utils.upload_data.UploadData.load_wellhead_measurement_data}\pysiglinewithargsret{\bfcode{load\_wellhead\_measurement\_data}}{\emph{csv\_pathname}}{}
Load wellhead measurement from a csv file into the database.
\begin{quote}\begin{description}
\item[{Parameters}] \leavevmode
\textbf{csv\_pathname} (\emph{str}) -- path to the csv file

\item[{Returns}] \leavevmode
message detailing number of records loaded.

\item[{Return type}] \leavevmode
str

\end{description}\end{quote}

\end{fulllineitems}

\index{load\_wellhead\_mp\_data() (webb\_utils.upload\_data.UploadData method)}

\begin{fulllineitems}
\phantomsection\label{modules:webb_utils.upload_data.UploadData.load_wellhead_mp_data}\pysiglinewithargsret{\bfcode{load\_wellhead\_mp\_data}}{\emph{csv\_pathname}}{}
Load wellhead mp data from a csv file into the database.
\begin{quote}\begin{description}
\item[{Parameters}] \leavevmode
\textbf{csv\_pathname} (\emph{str}) -- path to the csv file

\item[{Returns}] \leavevmode
message detailing number of records loaded.

\item[{Return type}] \leavevmode
str

\end{description}\end{quote}

\end{fulllineitems}

\index{load\_wslh\_anion\_data() (webb\_utils.upload\_data.UploadData method)}

\begin{fulllineitems}
\phantomsection\label{modules:webb_utils.upload_data.UploadData.load_wslh_anion_data}\pysiglinewithargsret{\bfcode{load\_wslh\_anion\_data}}{\emph{csv\_pathname}}{}
Load Wisconsin State Lab of Hygiene anion data from a csv file into the database.
\begin{quote}\begin{description}
\item[{Parameters}] \leavevmode
\textbf{csv\_pathname} (\emph{str}) -- path to the csv file

\item[{Returns}] \leavevmode
message detailing number of records loaded.

\item[{Return type}] \leavevmode
str

\end{description}\end{quote}

\end{fulllineitems}

\index{load\_wslh\_cation\_data() (webb\_utils.upload\_data.UploadData method)}

\begin{fulllineitems}
\phantomsection\label{modules:webb_utils.upload_data.UploadData.load_wslh_cation_data}\pysiglinewithargsret{\bfcode{load\_wslh\_cation\_data}}{\emph{csv\_pathname}}{}
Load Wisconsin State Lab of Hygiene cation data from a csv file into the database.
\begin{quote}\begin{description}
\item[{Parameters}] \leavevmode
\textbf{csv\_pathname} (\emph{str}) -- path to the csv file

\item[{Returns}] \leavevmode
message detailing number of records loaded.

\item[{Return type}] \leavevmode
str

\end{description}\end{quote}

\end{fulllineitems}

\index{load\_www\_sites\_data() (webb\_utils.upload\_data.UploadData method)}

\begin{fulllineitems}
\phantomsection\label{modules:webb_utils.upload_data.UploadData.load_www_sites_data}\pysiglinewithargsret{\bfcode{load\_www\_sites\_data}}{\emph{csv\_pathname}}{}
Load www site anion data from a csv file into the database.
\begin{quote}\begin{description}
\item[{Parameters}] \leavevmode
\textbf{csv\_pathname} (\emph{str}) -- path to the csv file

\item[{Returns}] \leavevmode
message detailing number of records loaded.

\item[{Return type}] \leavevmode
str

\end{description}\end{quote}

\end{fulllineitems}


\end{fulllineitems}



\section{DB Utils}
\label{modules:db-utils}
Module containing utilites for database access.
\index{create\_db\_filter\_str() (in module webb\_utils.db\_utils)}

\begin{fulllineitems}
\phantomsection\label{modules:webb_utils.db_utils.create_db_filter_str}\pysiglinewithargsret{\code{webb\_utils.db\_utils.}\bfcode{create\_db\_filter\_str}}{\emph{param\_list}}{}
Create an Oracle safe string from a Python iterable.
\begin{quote}\begin{description}
\item[{Parameters}] \leavevmode
\textbf{param\_list} -- list of strings

\item[{Returns}] \leavevmode
Oracle safe string

\item[{Return type}] \leavevmode
string

\end{description}\end{quote}

\end{fulllineitems}

\index{AlchemDB (class in webb\_utils.db\_utils)}

\begin{fulllineitems}
\phantomsection\label{modules:webb_utils.db_utils.AlchemDB}\pysiglinewithargsret{\strong{class }\code{webb\_utils.db\_utils.}\bfcode{AlchemDB}}{\emph{schema}, \emph{password}, \emph{db\_name}}{}
Create an Oracle session.
\begin{quote}\begin{description}
\item[{Parameters}] \leavevmode\begin{itemize}
\item {} 
\textbf{schema} (\emph{str}) -- schema user name

\item {} 
\textbf{password} (\emph{str}) -- schema user password

\item {} 
\textbf{db\_name} (\emph{str}) -- database name

\end{itemize}

\end{description}\end{quote}
\index{create\_session() (webb\_utils.db\_utils.AlchemDB method)}

\begin{fulllineitems}
\phantomsection\label{modules:webb_utils.db_utils.AlchemDB.create_session}\pysiglinewithargsret{\bfcode{create\_session}}{}{}
Create an Oracle session
\begin{quote}\begin{description}
\item[{Returns}] \leavevmode
Oracle database session

\item[{Return type}] \leavevmode
sqlalchemy.orm.session.Session

\end{description}\end{quote}

\end{fulllineitems}


\end{fulllineitems}



\chapter{Example Usage}
\label{example_usage:example-usage}\label{example_usage::doc}
Illustration of using this package to retrieve data through
a Python interactive console.


\section{Data Retrieval}
\label{example_usage:data-retrieval}
This example is providing assuming that one is in the
py-webb-db-utils has been installed as a package. These
examples show MS Excel 2007 exports being placed at
C:/Users/joe/downloads/, but can be any appropriate directory.

\begin{Verbatim}[commandchars=\\\{\}]
\PYG{g+gp}{\PYGZgt{}\PYGZgt{}\PYGZgt{} }\PYG{k+kn}{from} \PYG{n+nn}{webb\PYGZus{}utils.retrieve\PYGZus{}data} \PYG{k+kn}{import} \PYG{n}{RetrieveData}
\PYG{g+gp}{\PYGZgt{}\PYGZgt{}\PYGZgt{} }\PYG{n}{SCHEMA\PYGZus{}NAME} \PYG{o}{=} \PYG{l+s}{\PYGZsq{}}\PYG{l+s}{my\PYGZus{}schema}\PYG{l+s}{\PYGZsq{}}
\PYG{g+gp}{\PYGZgt{}\PYGZgt{}\PYGZgt{} }\PYG{n}{PASSWORD} \PYG{o}{=} \PYG{l+s}{\PYGZsq{}}\PYG{l+s}{my\PYGZus{}password}\PYG{l+s}{\PYGZsq{}}
\PYG{g+gp}{\PYGZgt{}\PYGZgt{}\PYGZgt{} }\PYG{n}{DB\PYGZus{}NAME} \PYG{o}{=} \PYG{l+s}{\PYGZsq{}}\PYG{l+s}{db.name.blah}\PYG{l+s}{\PYGZsq{}}
\PYG{g+gp}{\PYGZgt{}\PYGZgt{}\PYGZgt{} }\PYG{n}{START\PYGZus{}DATE} \PYG{o}{=} \PYG{l+s}{\PYGZsq{}}\PYG{l+s}{01\PYGZhy{}OCT\PYGZhy{}2006}\PYG{l+s}{\PYGZsq{}}
\PYG{g+gp}{\PYGZgt{}\PYGZgt{}\PYGZgt{} }\PYG{n}{END\PYGZus{}DATE} \PYG{o}{=} \PYG{l+s}{\PYGZsq{}}\PYG{l+s}{30\PYGZhy{}SEP\PYGZhy{}2007}\PYG{l+s}{\PYGZsq{}}
\PYG{g+gp}{\PYGZgt{}\PYGZgt{}\PYGZgt{} }\PYG{n}{rd} \PYG{o}{=} \PYG{n}{RetrieveData}\PYG{p}{(}\PYG{n}{SCHEMA\PYGZus{}NAME}\PYG{p}{,} \PYG{n}{PASSWORD}\PYG{p}{,} \PYG{n}{DB\PYGZus{}NAME}\PYG{p}{)}
\PYG{g+gp}{\PYGZgt{}\PYGZgt{}\PYGZgt{} }\PYG{c}{\PYGZsh{} get well datums}
\PYG{g+gp}{\PYGZgt{}\PYGZgt{}\PYGZgt{} }\PYG{n}{gwd} \PYG{o}{=} \PYG{n}{rd}\PYG{o}{.}\PYG{n}{get\PYGZus{}well\PYGZus{}datums}\PYG{p}{(}\PYG{n}{excel\PYGZus{}export\PYGZus{}path}\PYG{o}{=}\PYG{l+s}{\PYGZsq{}}\PYG{l+s}{C:/Users/joe/downloads/well\PYGZus{}datums.xlsx}\PYG{l+s}{\PYGZsq{}}\PYG{p}{)}
\PYG{g+gp}{\PYGZgt{}\PYGZgt{}\PYGZgt{} }\PYG{c}{\PYGZsh{} get well uv}
\PYG{g+gp}{\PYGZgt{}\PYGZgt{}\PYGZgt{} }\PYG{n}{gwu} \PYG{o}{=} \PYG{n}{rd}\PYG{o}{.}\PYG{n}{get\PYGZus{}well\PYGZus{}uvs}\PYG{p}{(}\PYG{n}{START\PYGZus{}DATE}\PYG{p}{,} \PYG{n}{END\PYGZus{}DATE}\PYG{p}{,}
\PYG{g+go}{        excel\PYGZus{}export\PYGZus{}path=\PYGZsq{}C:/Users/joe/downloads/well\PYGZus{}uvs.xlsx\PYGZsq{})}
\PYG{g+gp}{\PYGZgt{}\PYGZgt{}\PYGZgt{} }\PYG{c}{\PYGZsh{} get carbon data}
\PYG{g+gp}{\PYGZgt{}\PYGZgt{}\PYGZgt{} }\PYG{n}{gcd} \PYG{o}{=} \PYG{n}{rd}\PYG{o}{.}\PYG{n}{get\PYGZus{}carbon\PYGZus{}data}\PYG{p}{(}\PYG{n}{START\PYGZus{}DATE}\PYG{p}{,} \PYG{n}{END\PYGZus{}DATE}\PYG{p}{,}
\PYG{g+go}{        excel\PYGZus{}export\PYGZus{}path=\PYGZsq{}C:/Users/joe/downloads/carbon\PYGZus{}data.xlsx\PYGZsq{})}
\PYG{g+gp}{\PYGZgt{}\PYGZgt{}\PYGZgt{} }\PYG{c}{\PYGZsh{} get alkalinity}
\PYG{g+gp}{\PYGZgt{}\PYGZgt{}\PYGZgt{} }\PYG{n}{gda} \PYG{o}{=} \PYG{n}{rd}\PYG{o}{.}\PYG{n}{get\PYGZus{}data\PYGZus{}with\PYGZus{}alkalinity}\PYG{p}{(}\PYG{n}{START\PYGZus{}DATE}\PYG{p}{,} \PYG{n}{END\PYGZus{}DATE}\PYG{p}{,}
\PYG{g+go}{        excel\PYGZus{}export\PYGZus{}path=\PYGZsq{}C:/Users/joe/downloads/data\PYGZus{}alk.xlsx\PYGZsq{})}
\PYG{g+gp}{\PYGZgt{}\PYGZgt{}\PYGZgt{} }\PYG{c}{\PYGZsh{} get check values}
\PYG{g+gp}{\PYGZgt{}\PYGZgt{}\PYGZgt{} }\PYG{n}{gwcv} \PYG{o}{=} \PYG{n}{rd}\PYG{o}{.}\PYG{n}{get\PYGZus{}well\PYGZus{}check\PYGZus{}values}\PYG{p}{(}\PYG{n}{START\PYGZus{}DATE}\PYG{p}{,} \PYG{n}{END\PYGZus{}DATE}\PYG{p}{,}
\PYG{g+go}{        excel\PYGZus{}export\PYGZus{}path=\PYGZsq{}C:/Users/joe/downloads/check\PYGZus{}values.xlsx\PYGZsq{})}
\PYG{g+gp}{\PYGZgt{}\PYGZgt{}\PYGZgt{} }\PYG{c}{\PYGZsh{} get piezo sites}
\PYG{g+gp}{\PYGZgt{}\PYGZgt{}\PYGZgt{} }\PYG{n}{gps} \PYG{o}{=} \PYG{n}{rd}\PYG{o}{.}\PYG{n}{get\PYGZus{}piezo\PYGZus{}sites}\PYG{p}{(}\PYG{n}{START\PYGZus{}DATE}\PYG{p}{,} \PYG{n}{END\PYGZus{}DATE}\PYG{p}{,}
\PYG{g+go}{        excel\PYGZus{}export\PYGZus{}path=\PYGZsq{}C:/Users/joe/downloads/piezo\PYGZus{}sites.xlsx\PYGZsq{})}
\PYG{g+gp}{\PYGZgt{}\PYGZgt{}\PYGZgt{} }\PYG{c}{\PYGZsh{} get site information}
\PYG{g+gp}{\PYGZgt{}\PYGZgt{}\PYGZgt{} }\PYG{n}{sites} \PYG{o}{=} \PYG{n}{rd}\PYG{o}{.}\PYG{n}{get\PYGZus{}site\PYGZus{}info}\PYG{p}{(}\PYG{n}{excel\PYGZus{}export\PYGZus{}path}\PYG{o}{=}\PYG{l+s}{\PYGZsq{}}\PYG{l+s}{C:/Users/joe/downloads/site\PYGZus{}info.xlsx}\PYG{l+s}{\PYGZsq{}}\PYG{p}{)}
\PYG{g+gp}{\PYGZgt{}\PYGZgt{}\PYGZgt{} }\PYG{c}{\PYGZsh{} close the session}
\PYG{g+gp}{\PYGZgt{}\PYGZgt{}\PYGZgt{} }\PYG{n}{rd}\PYG{o}{.}\PYG{n}{close\PYGZus{}session}\PYG{p}{(}\PYG{p}{)}
\end{Verbatim}


\section{Data Entry}
\label{example_usage:data-entry}
This example demonstrates the use this package to load
data from a CSV into the database. The files for this demonstration
can be found in the example\_upload\_files directory of this project.

A salient feature to not about the upload files, is that the
order of the columns mirrors the order of the columns specified
in webb\_utils.db\_mappings.upload\_columns. This is important as
it is necessary for this package to parse the CSV file and then
load the data into the correct database table columns.

Let's say that there is a new sample that needs to be added
to the database. The CSV for this is called ``sample\_load.csv''
and for demonstration purposes, say this file is
currently located at `C:/Users/anna/Documents/sample\_load.csv'.
Loading this data via a Python interactive console would proceed as follows:

\begin{Verbatim}[commandchars=\\\{\}]
\PYG{g+gp}{\PYGZgt{}\PYGZgt{}\PYGZgt{} }\PYG{k+kn}{from} \PYG{n+nn}{webb\PYGZus{}utils.upload\PYGZus{}data} \PYG{k+kn}{import} \PYG{n}{UploadData}
\PYG{g+gp}{\PYGZgt{}\PYGZgt{}\PYGZgt{} }\PYG{n}{SCHEMA\PYGZus{}NAME} \PYG{o}{=} \PYG{l+s}{\PYGZsq{}}\PYG{l+s}{my\PYGZus{}schema}\PYG{l+s}{\PYGZsq{}}
\PYG{g+gp}{\PYGZgt{}\PYGZgt{}\PYGZgt{} }\PYG{n}{PASSWORD} \PYG{o}{=} \PYG{l+s}{\PYGZsq{}}\PYG{l+s}{my\PYGZus{}password}\PYG{l+s}{\PYGZsq{}}
\PYG{g+gp}{\PYGZgt{}\PYGZgt{}\PYGZgt{} }\PYG{n}{DB\PYGZus{}NAME} \PYG{o}{=} \PYG{l+s}{\PYGZsq{}}\PYG{l+s}{db.name.blah}\PYG{l+s}{\PYGZsq{}}
\PYG{g+gp}{\PYGZgt{}\PYGZgt{}\PYGZgt{} }\PYG{n}{ud} \PYG{o}{=} \PYG{n}{UploadData}\PYG{p}{(}\PYG{n}{SCHEMA\PYGZus{}NAME}\PYG{p}{,} \PYG{n}{PASSWORD}\PYG{p}{,} \PYG{n}{DB\PYGZus{}NAME}\PYG{p}{)} \PYG{c}{\PYGZsh{} this creates a database session}
\PYG{g+gp}{\PYGZgt{}\PYGZgt{}\PYGZgt{} }\PYG{c}{\PYGZsh{} load and commit the sample data into the database}
\PYG{g+gp}{\PYGZgt{}\PYGZgt{}\PYGZgt{} }\PYG{n}{sample\PYGZus{}upload} \PYG{o}{=} \PYG{n}{ud}\PYG{o}{.}\PYG{n}{load\PYGZus{}sample\PYGZus{}data}\PYG{p}{(}\PYG{l+s}{\PYGZsq{}}\PYG{l+s}{C:/Users/anna/Documents/sample\PYGZus{}load.csv}\PYG{l+s}{\PYGZsq{}}\PYG{p}{)}
\end{Verbatim}

In addition, loading that sample data, there is also strontium
isotope data that needs to be loaded for the sample. The data file
from the lab is `anion\_load.csv' and that this file has been placed at
`C:/Users/anna/Documents/strontium\_load.csv'. This anion data can be loaded
as follows:

\begin{Verbatim}[commandchars=\\\{\}]
\PYG{g+gp}{\PYGZgt{}\PYGZgt{}\PYGZgt{} }\PYG{c}{\PYGZsh{} load and commit the anion data into the database}
\PYG{g+gp}{\PYGZgt{}\PYGZgt{}\PYGZgt{} }\PYG{n}{anion\PYGZus{}load} \PYG{o}{=} \PYG{n}{ud}\PYG{o}{.}\PYG{n}{load\PYGZus{}strontium\PYGZus{}isotope\PYGZus{}data}\PYG{p}{(}\PYG{l+s}{\PYGZsq{}}\PYG{l+s}{C:/Users/anna/Documents/strontium\PYGZus{}load.csv}\PYG{l+s}{\PYGZsq{}}\PYG{p}{)}
\end{Verbatim}

If those are the only two things we want to load,
the database session needs to be ended. This can be
done as follows:

\begin{Verbatim}[commandchars=\\\{\}]
\PYG{g+gp}{\PYGZgt{}\PYGZgt{}\PYGZgt{} }\PYG{n}{ud}\PYG{o}{.}\PYG{n}{close\PYGZus{}session}\PYG{p}{(}\PYG{p}{)}
\end{Verbatim}


\chapter{Indices and tables}
\label{webb_utils_doc:indices-and-tables}\begin{itemize}
\item {} 
\emph{genindex}

\item {} 
\emph{modindex}

\item {} 
\emph{search}

\end{itemize}



\renewcommand{\indexname}{Index}
\printindex
\end{document}
